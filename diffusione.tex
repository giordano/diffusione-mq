\documentclass[a4paper,fleqn,twoside,12pt]{article}
\usepackage[T1]{fontenc}
\usepackage[light,slantedGreeks]{kpfonts}
\usepackage[utf8]{inputenc}
\usepackage[italian]{babel}

%%%%% Pacchetti caricati
\usepackage{layaureo}
\usepackage[autostyle=true]{csquotes}
\usepackage[style=numeric,hyperref,abbreviate=false,backend=biber]{biblatex}
\usepackage[font=small,format=hang,labelfont=bf]{caption}
\usepackage{tikz}
% `mathtools' serve per definire la norma e il valore assoluto
\usepackage{mathtools}
% `bm' serve per scrivere i vettori in corsivo con il comando
% `\bm{vettore}'. Deve sostituire il comando `\mathbf{vettore}' perché questo
% restituisce erroneamente lettere in tondo, non in corsivo e non funziona con
% le lettere greche.
\usepackage{bm}
\usepackage{hyperref}

% impostazioni per il pacchetto `hyperref'. Per l'elenco di tutte le opzioni del
% documento consulta il manuale: `texdoc hyperref'
\hypersetup{
  pdftitle={Introduzione alla teoria della diffusione in meccanica quantistica},
  pdfauthor={Mosè Giordano},
  breaklinks=true,% permette di spezzare i link su più righe
  bookmarksnumbered,% inserisce i numeri delle sezioni nei segnalibri
  hidelinks % link neri e senza bordi colorati, adatto per la stampa
}

\addbibresource{bibliografia.bib} % nome del file contenente la bibliografia

%%%%% Comandi personalizzati
% ridefinisco i comandi per alcune lettere greche in modo che si usino le
% varianti
\renewcommand{\phi}{\varphi}
\renewcommand{\epsilon}{\varepsilon}

% comando per evidenziare le sezioni da completare o espandere.  NOTA: è un
% comando provvisorio, serve solo durante la scrittura del testo, ricordarsi di
% eliminarlo quando non serve più.
\newcommand{\completare}[1]{\textcolor{red}{#1}}

% Operatori
\newcommand*{\dd}{\mathop{}\!\mathrm{d}} % Operatore differenziale \dd
\DeclareMathOperator{\e}{\mathrm{e}} % Numero di Eulero
\DeclareMathOperator{\uimm}{\mathrm{i}} % unità immaginaria
% Operatore valore assoluto \abs{x}. Usa \abs*{} per le frazioni
\DeclarePairedDelimiter{\abs}{\lvert}{\rvert}
% Operatore norma \norm{x}. Usa \norm*{} per le frazioni
\DeclarePairedDelimiter{\norm}{\lVert}{\rVert}

%% Derivate
% Derivata totale: \toder[ordine]{funzione}{variabile}
\newcommand*{\toder}[3][]{\frac{{\dd^{#1}}#2}{\dd {#3}^{#1}}}
% Derivata parziale \parder[ordine]{funzione}{variabile}
% Per la definizione del comando `parder' (per inserire le derivate parziali)
% vedi
% http://www.guitex.org/home/index.php?option=com_kunena&func=view&catid=5&id=42178&Itemid=60#42199
\makeatletter
\newcommand{\parder}[2]{\begingroup
  \@tempswafalse\toks@={}\count@=\z@
  \@for\next:=#2\do
    {\expandafter\check@var\next\@nil
     \advance\count@\parder@exp
     \if@tempswa
       \toks@=\expandafter{\the\toks@\,}%
     \else
       \@tempswatrue
     \fi
     \toks@=\expandafter{\the\expandafter\toks@\expandafter\partial\parder@var}}%
  \frac{\partial\ifnum\count@=\@ne\else^{\number\count@}\fi#1}{\the\toks@}%
  \endgroup}
\def\check@var{\@ifstar{\mult@var}{\one@var}}
\def\mult@var#1#2\@nil{\def\parder@var{#2^{#1}}\def\parder@exp{#1}}
\def\one@var#1\@nil{\def\parder@var{#1}\chardef\parder@exp\@ne}
\makeatother

% Derivate per le formule in linea (usare \frac in linea è eccessivo). La `l'
% iniziale nel nome distingue questi comandi da quelli per le formule fuori
% corpo. Non uso `\dd' ma `\mathrm{d}' perché nelle formule in linea `\dd'
% aggiunge una spaziatura non adatta. Non sono dei comandi bellissimi, ma
% permettono di passare facilmente da formula in linea a fuori corpo e viceversa
% cambiando una lettera.
% Derivata totale: \ltoder[ordine]{funzione}{variabile}
\newcommand*{\ltoder}[3][]{\mathrm{d}^{#1}#2 / \mathrm{d} {#3}^{#1}}
% Derivata parziale: \lparder[ordine]{funzione}{variabile}
\newcommand*{\lparder}[3][]{\partial^{#1} #2 / \partial {#3}^{#1}}
% NOTA: `\parder' e `\lparder' non sono completamente interscambiabili, il primo
% comando è molto più complesso e permette di inserire le derivate miste, a
% differenza del secondo.

% Versore. Esempi: versore x: `\versore{x}', versore i: \versor{\imath}, versore
% j: \versor{\jmath} (solo `i' e `j' richiedono `\imath' e `\jmath', altrimenti
% il puntino litiga con `\hat')
\newcommand*{\versor}[1]{\hat{\bm{#1}}}


\title{Introduzione alla teoria della diffusione \\ in meccanica quantistica}
\author{Mosè Giordano}

\begin{document}
\maketitle
\tableofcontents

\section{\completare{Concetti generali}}
\label{sec:concetti-generali}

Flusso di proiettili per unità di area e tempo
\begin{equation}
  J_{a} = \frac{\text{numero proiettili}}{\text{area}\cdot\text{tempo}} =
  \frac{n_{a}}{S_{\textup{t}} T}
\end{equation}
$S_{\textup{t}} = \text{sezione trasversale}$, $T = \text{tempo}$.  Numero di
processi di diffusione per unità di tempo
\begin{equation}
  n_{t} = \sigma_{\textup{tot}} J_{a} N_{a}
\end{equation}
$N_{A} = \text{numero di bersagli $A$}$,
$\sigma_{\textup{tot}} = \text{\emph{sezione d'urto totale}}$.

Numero $\dd n_{b}$ di particelle diffuse per unità di tempo nell'angolo solido
$\dd\Omega_{b}$.  È intuitivo capire che $\dd n_{b}$ deve essere proporzionale
al flusso $J_{a}$ di proiettili e all'angolo solido $\dd\Omega_{b}$ in cui si
vanno a rilevare le particelle diffuse.  Indichiamo con $\sigma(\theta,\phi)$ il
coefficiente di proporzionalità
\begin{equation}
  \dd n_{b} = \frac{\text{numero di eiettili in $\dd\Omega_{b}$}}{T} = J_{a}
  \dd\Omega_{b} \sigma(\theta,\phi).
\end{equation}
La quantità $\sigma(\theta,\phi)$ prende il nome di
\emph{sezione d'urto differenziale}.  Dividendo $\dd n_{b}$ per l'area
$S_{\textup{d}}$ del rivelatore otteniamo il flusso $J_{b}$ di particelle
diffuse per unità di tempo e di area.  L'area $S_{\textup{d}}$ del rivelatore è
data dal prodotto fra il quadrato della distanza $r$ in cui si trova il
rivelatore e l'apertura angolare $\dd\Omega_{b}$ del rivelatore:
$S_{\textup{d}} = r^{2}\dd\Omega_{b}$.  Allora
\begin{equation}
  J_{b} = \frac{\dd n_{b}}{S_{\textup{d}}} = \frac{J_{a} \sigma(\theta,\phi)
    \dd\Omega_{b}}{r^{2} \dd\Omega_{b}} = \frac{J_{a}
    \sigma(\theta,\phi)}{r^{2}},
\end{equation}
da cui ricaviamo che la sezione d'urto differenziale è data da
\begin{equation}
  \label{eq:sezione-d'urto}
  \sigma(\theta,\phi) = \frac{J_{b}r^{2}}{J_{a}}.
\end{equation}
La sezione d'urto differenziale dipende anche dall'energia del fascio di
proiettili e dallo specifico canale di diffusione seguito.  Nota
$\sigma(\theta,\phi)$, la sezione d'urto totale si ottiene integrando su tutto
l'angolo solido
\begin{equation}
  \sigma_{\textup{tot}} = \int \sigma(\theta,\phi) \dd\Omega = \int_{0}^{2\pi}
  \dd \phi \int_{0}^{\pi} \sin \theta \dd\theta \sigma(\theta,\phi).
\end{equation}

\subsection{Sistemi di riferimento del laboratorio e del centro di massa}
\label{sec:sistemi-riferimento}

\section{Diffusione in meccanica quantistica}
\label{sec:meccanica-quantistica}

Tutto quello che abbiamo detto finora è valido in generale.  Introdurremo ora la
teoria della diffusione nell'ambito della meccanica quantistica non
relativistica.  Prima di procedere precisiamo le ipotesi sotto le quali
affronteremo il problema
\begin{itemize}
\item supponiamo che le particelle coinvolte nella diffusione siano prive di
  spin;
\item non consideriamo la struttura interna delle particelle.  In questo modo
  escludiamo dalla trattazione le diffusioni anelastiche e ci occuperemo solo di
  quelle elastiche;
\item supponiamo che il bersaglio è sufficientemente piccolo da poter trascurare
  processi di diffusione multipla;
\item trascuriamo la possibilità di coerenza fra onde diffuse dalle differenti
  particelle che costituiscono il bersaglio;
\item supponiamo che l'interazione fra proiettile e bersaglio sia descritta da
  un potenziale $V$ dipendente dalla distanza relativa fra le due particelle
  $\bm{r}_{1} - \bm{r}_{2}$: $V = V(\bm{r}_{1} - \bm{r}_{2})$.  Questo ci
  permetterà di adottare il formalismo ben noto del problema dei due corpi.
\end{itemize}
Abbiamo fatto queste assunzioni non perché non siano possibili casi differenti
(sono per esempio molto importanti i casi di diffusione di particelle dotate di
spin e di diffusioni inelastiche) ma solo per semplificare la presente
trattazione.

L'hamiltoniana per un sistema di due particelle senza spin interagenti fra loro
è
\begin{equation}
  H = -\frac{\hslash^{2}}{2m_{1}} \nabla_{1}^{2}
  -\frac{\hslash^{2}}{2m_{2}}\nabla_{2}^{2} + V(\bm{r_{1}} - \bm{r}_{2}).
\end{equation}
Come anticipato, tratteremo il problema come il solito problema dei due corpi,
quindi definiamo la posizione $\bm{R}$ del centro di massa e la posizione
relativa $\bm{r}$ rispettivamente come
\begin{subequations}
  \begin{align}
    \bm{R} &= \frac{m_{1}\bm{r}_{1} + m_{2}\bm{r}_{2}}{m_{1} + m_{2}}, \\
    \bm{r} &= \bm{r_{1}} - \bm{r}_{2}.
  \end{align}
\end{subequations}
In questo modo l'hamiltoniana assume la forma più semplice
\begin{equation}
  \label{eq:hamiltoniana}
  H = -\frac{\hslash^{2}}{2(m_{1} + m_{2})}\nabla_{\bm{R}}^{2} -
  \frac{\hslash^{2}}{2m}\nabla_{\bm{r}}^{2} + V(\bm{r}).
\end{equation}
Il primo termine è l'energia cinetica del centro di massa, gli ultimi due sono
l'energia della particella fittizia di massa ridotta
$m = m_{1}m_{2}/(m_{1}+m_{2})$.  Vogliamo determinare gli autostati stazionari
dell'hamiltoniana e in particolare con valori positivi dell'energia, associati a
stati non legati, perché se lo stato delle due particelle fosse legato sarebbe
nulla la probabilità di trovarle a distanza reciproca infinita.  Come noto,
possiamo cercare autostati dell'hamiltoniana~\eqref{eq:hamiltoniana} della forma
separabile $\Psi(\bm{R}, \bm{r}) = \Phi(\bm{R})\psi(\bm{r})$.  In particolare,
$\psi(\bm{r})$ deve soddisfare la seguente equazione di Schröedinger stazionaria
\begin{equation}
  \bigg(-\frac{\hslash^{2}}{2m}\nabla^{2} + V(\bm{r})\bigg)\psi(\bm{r}) =
  E\psi(\bm{r}).
\end{equation}
Il moto del centro di massa non è di nostro interesse poiché la sua hamiltoniana
è semplicemente quella di particella libera, dunque ci occuperemo di studiare
solo il moto relativo fra la particella incidente e il bersaglio.

\subsection{Forma asintotica degli stati stazionari}
\label{sec:forma-asintintotica}

Nel paragrafo~\ref{sec:concetti-generali} abbiamo visto che nei problemi di
diffusione è presente un flusso $J_{a}$ incidente sul bersaglio e un flusso
$J_{b}$ diffuso dal bersaglio e che si allontana in tutte le direzioni.  Ci
aspettiamo dunque che la funzione d'onda $\psi(\bm{r})$ sia la somma di una
funzione d'onda incidente $\psi_{a}$, che rappresenta il flusso di proiettili, e
una funzione d'onda di diffusione $\psi_{b}$, che rappresenta gli eiettili
\begin{equation}
  \psi(\bm{r}) = \psi_{a}(\bm{r}) + \psi_{b}(\bm{r}).
\end{equation}
Molto tempo prima di raggiungere il bersaglio, il proiettile si muove come una
particella libera perché per valori sufficientemente grandi della distanza
relativa $r$ il potenziale $V(\bm{r})$ è praticamente nullo.  Quindi la
componente $\psi_{a}(\bm{r})$ sarà un'onda piana del tipo
$\e^{\uimm \bm{k}\cdot\bm{r}}$, con $\bm{k}$ vettore d'onda associato alla
particella di massa $m$, il cui modulo è legato all'energia $E$ dalla nota
relazione $E = \hslash^{2}k^{2}/(2m)$.  Nella vicinanza del bersaglio, la
funzione d'onda della particella subirà una profonda modifica a causa
dell'interazione con il bersaglio.  Tuttavia molto tempo dopo l'interazione,
l'eiettile sarà lontano dall'influenza del potenziale e la sua funzione d'onda
avrà raggiunto una forma ``stabile''.  Facendo un'analogia con la meccanica
ondulatoria classica possiamo prevedere una forma esplicita per il comportamento
asintotico di $\psi_{b}(\bm{r})$.  Un'onda piana che urta contro un ostacolo,
cioè il bersaglio nel problema di diffusione, si diffonde in tutto lo spazio
come un'onda sferica uscente del tipo $f(\theta,\phi)\e^{\uimm kr}/r$, con
$f(\theta,\phi)$ ampiezza dell'onda di diffusione.  Essa dipende dalle due
coordinate angolari sferiche $\theta$ e $\phi$ perché la diffusione non è in
generale isotropica.  Il fattore $1/r$ assicura che il flusso della densità di
probabilità $\abs{\psi_{b}}^{2}$ sia costante per ogni superficie sferica
centrata nel bersaglio.  Il comportamento asintotico della funzione d'onda
$\psi(\bm{r})$ sarà del tipo
\begin{equation}
  \label{eq:forma-asintotica}
  \psi(\bm{r}) = \psi_{a}(\bm{r}) + \psi_{b}(\bm{r}) =A \bigg( \e^{\uimm
    \bm{k}\cdot\bm{r}} + f(\theta,\phi)\frac{\e^{\uimm kr}}{r} \bigg),
\end{equation}
con $A$ fattore di normalizzazione.  Nei paragrafi successivi vedremo che la
forma asintotica~\eqref{eq:forma-asintotica} della funzione d'onda in un
problema di diffusione, ricavata qui in base a un'analogia classica, è ben
giustificata sotto opportune ipotesi.

\subsection{Calcolo della sezione d'urto}
\label{sec:sez-urto-mq}

Lo stato quantico $\psi(\bm{r})$ di una particella non descrive esattamente la
sua posizione ma l'ampiezza della probabilità di trovare la particella nella
posizione $\bm{r}$.  Analogamente, il flusso che dobbiamo considerare per
calcolare la sezione d'urto differenziale è il flusso di probabilità, cioè la
probabilità per unità di tempo che la particella attraversi l'area unitaria.  È
noto che per una particella di massa $m$ nello stato $\psi$, il flusso di
probabilità vale
\begin{equation}
  \bm{J} = \frac{\hslash}{m} \Im(\psi^{*}\nabla\psi).
\end{equation}
Applicando questa equazione alla componente incidente della funzione d'onda
asintotica~\eqref{eq:forma-asintotica} abbiamo
\begin{equation}
  \bm{J}_{a} = \frac{\hslash}{m}\Im(\psi_{a}^{*}\nabla\psi_{a}) =
  \frac{\abs{A}^{2}\hslash \bm{k}}{m}.
\end{equation}
Per la componente di diffusione abbiamo
\begin{subequations}
  \begin{align}
    (\bm{J}_{b})_{r} &=
    \frac{\hslash}{m}\Im\bigg(\psi_{b}^{*}\parder{\psi_{b}}{r}\bigg) =
    \abs{Af(\theta,\phi)}^{2} \frac{\hslash k}{mr^{2}}, \\
    (\bm{J}_{b})_{\theta} &= \frac{\hslash}{m} \frac{1}{r^{3}} \Re
    \bigg(\frac{1}{\uimm}
    f^{*}(\theta,\phi) \parder{}{\theta}f(\theta,\phi)\bigg), \\
    (\bm{J}_{b})_{\phi} &= \frac{\hslash}{m} \frac{1}{r^{3} \sin\theta}
    \Re\bigg(\frac{1}{i} f^{*}(\theta,\phi) \parder{}{\phi}f(\theta,\phi)\bigg).
  \end{align}
\end{subequations}
Poiché stiamo considerando il comportamento asintotico, grandi $r$, le
componenti angolari del flusso $(\bm{J}_{b})_{\theta}$ e $(\bm{J}_{b})_{\phi}$
sono trascurabili rispetto alla componente radiale $(\bm{J}_{b})_{\textup{r}}$,
allora approssimiamo $J_{b} \approx (\bm{J}_{b})_{r}$.  Inserendo questi
risultati nell'equazione~\eqref{eq:sezione-d'urto} troviamo che la sezione
d'urto differenziale è
\begin{equation}
  \sigma(\theta,\phi) = \frac{J_{b}r^{2}}{J_{a}} = \abs{f(\theta,\phi)}^{2}.
\end{equation}
La sezione d'urto differenziale è la quantità di maggior interesse nei problemi
di diffusione perché è quella misurabile sperimentalmente e abbiamo visto che è
uguale al modulo quadro di $f(\theta,\phi)$, chiamata per questo motivo
\emph{ampiezza di diffusione}.  La sezione d'urto non dipende dal fattore di
normalizzazione $A$ e spesso nel seguito lo trascureremo ponendolo uguale a $1$.
Nei due prossimi paragrafi studieremo due metodi alternativi per calcolare
l'ampiezza di diffusione: lo sviluppo in onde parziali e l'approssimazione di
Born.

\section{Sviluppo in onde parziali}
\label{sec:onde-parziali}

Il primo metodo che studieremo è particolarmente utile nei casi in cui il
potenziale di interazione ha simmetria sferica, cioè dipende solo dal modulo $r$
della distanza relativa fra i due corpi: $V(\bm{r}) = V(r)$.  L'equazione di
Schröedinger stazionaria assume la forma
\begin{equation}
  (\nabla^{2} + k^{2} - U(r)) \psi(\bm{r}) = 0,
\end{equation}
con $U(r) = (2m/\hslash^{2})V(r)$.  In questo caso sappiamo che una soluzione
dell'equazione precedente è del tipo separabile
$R_{kl}(r)Y_{l}^{m}(\theta,\phi)$, in cui $Y_{l}^{m}(\theta,\phi))$ è
un'armonica sferica e $R_{kl}(r)$ è una funzione puramente radiale che può
essere posta nella forma $R_{kl}(r) = u_{kl}(r)/r$ con la condizione
$u_{kl}(0) = 0$.  La soluzione generale dell'equazione precedente, autostato
dell'hamiltoniana con autovalore di energia $E = \hslash^{2}k^{2}/(2m)$, sarà
una combinazione lineare delle soluzioni appena illustrate con somma su tutti i
possibili valori di momento angolare $l$ e terza componente $m$, ma con fissato
valore del numero d'onda $k$ poiché stiamo considerando gli stati stazionari
dell'hamiltoniana
\begin{equation}
  \label{eq:onde-parziali1}
  \psi(\bm{r}) = \sum_{l = 0}^{+\infty} \sum_{m = -l}^{l} \psi_{klm}(\bm{r} =
  \sum_{l = 0}^{+\infty} \sum_{m = -l}^{l} a_{lm}
  \frac{u_{kl}(r)}{r}Y_{l}^{m}(\theta,\phi).
\end{equation}
Ciascuna delle funzioni $\psi_{klm}$ prende il nome di \emph{onda parziale},
quindi la loro combinazione lineare è detta \emph{sviluppo in onde parziali}.
L'equazione radiale che soddisfa $R_{kl}$ è
\begin{equation}
  \label{eq:diff-R-op}
  \bigg(\frac{1}{r^{2}}\toder{}{r}r^{2}\toder{}{r} + k^{2} - U(r) -
  \frac{l(l+1)}{r^{2}}\bigg)R_{kl}(r) = 0.
\end{equation}
Sostituendo $R_{kl}=u_{kl}/r$ troviamo che la funzione $u_{kl}$ soddisfa
un'equazione differenziale più semplice
\begin{equation}
  \bigg(\toder[2]{}{r} + k^{2} - U(r) - \frac{l(l+1)}{r^{2}}\bigg)u_{kl}(r) = 0.
\end{equation}
Il potenziale di interazione è a simmetria sferica, la particella incidente
rompe la completa simmetria definendo una direzione precisa che identifichiamo
con l'asse $\versor{z}$ (cioè $\bm{k} = k\versor{z}$), tuttavia non è presente
alcuna dipendenza dall'angolo azimutale $\phi$ e ci sarà pertanto simmetria
cilindrica.  Per annullare la dipendenza della funzione d'onda $\psi(\bm{r})$ da
$\phi$ nello sviluppo~\eqref{eq:onde-parziali1} dobbiamo considerare solo i
termini con $m = 0$ perché le armoniche sferiche dipende da $\phi$ attraverso
$\e^{\uimm m \phi}$.

% TODO: dimostrare lo sviluppo in onde parziali dell'onda piana
Tenendo anche presente quanto appena notato, possiamo ipotizzare una forma più
precisa per lo sviluppo in onde parziali di $\psi(\bm{r})$.  Partiamo
dall'osservare che un'onda piana, quindi in assenza di potenziale, può essere
sviluppata in onde parziali nel seguente modo
\begin{equation}
  \label{eq:sviluppo-onda-piana}
  \e^{\uimm \bm{k}\cdot\bm{r}} = \sum_{l} (2l+1) \uimm^{l} j_{l}(kr)
  P_{l}(\cos\theta),
\end{equation}
in cui $j_{l}$ è la funzione di Bessel sferica di ordine $l$, $P_{l}$ è il
polinomio di Legendre di grado $l$ e $\theta$ è l'angolo compreso fra
$\bm{k} = k\versor{z}$ e $\bm{r}$.  In presenza di un potenziale a corto raggio
possiamo supporre che l'andamento asintotico della funzione d'onda di diffusione
stazionaria, quindi con fissato valore dell'energia e di conseguenza di $k$, sia
del tipo
\begin{equation}
  \label{eq:onde-parziali2}
  \psi(\bm{r}) = \sum_{l} (2l+1) \uimm^{l} A_{l} R_{kl}(r) P_{l}(\cos\theta),
\end{equation}
Rispetto al caso di potenziale nullo stiamo dunque assumendo che al posto delle
funzioni di Bessel sferiche ci siano le funzioni $R_{kl}$ precedentemente
introdotte e inoltre dei coefficienti $A_{l}$ da determinare.

\subsection{Sfasamenti}
\label{sec:sfasamenti}

\begin{figure}
  \centering
  \begin{tikzpicture}[scale=0.7,font=\footnotesize]
    \draw[fill=white!90!black](0,0) circle (4);
    \node[circle,draw,align=center,fill=white!80!black] (0,0)
    {Regione di\\diffusione\\ $V \neq 0$};
    \node[align=center] at (0,2.6) {Regione intermedia\\ $V\approx0$};
    \node[align=center] at (10,1.5) {Zona di radiazione\\ $kr \gg 1$};
  \end{tikzpicture}
  \caption{Diffusione da potenziale a corto raggio con simmetria sferica.  Nelle
    immediate vicinanze del bersaglio, $r \approx 0$, si ha la regione di
    diffusione nella quale il potenziale non è trascurabile.  All'aumentare
    della distanza dal bersaglio il potenziale centrifugo
    $\hslash^2 l(l+1)/(2mr^2)$ domina su $V(r)$ e questa condizione determina la
    regione intermedia.  La zona di radiazione si trova a grande distanza dal
    bersaglio, quindi $kr \gg 1$ e sia il potenziale di interazione sia il
    potenziale centrifugo sono trascurabili rispetto a $k^2$.}
\label{fig:regioni-potenziale-sferico}
\end{figure}
Supponiamo che il potenziale di interazione sia a \emph{rapida decrescenza}, o
\emph{a corto raggio}, vale a dire per $r$ tendente all'infinito va a $0$ più
rapidamente di $1/r^{2}$
\begin{equation}
  \lim_{r \to \infty} r^{2}V(r) \in \mathbb{R}.
\end{equation}
In questo modo stiamo escludendo dalla trattazione il potenziale di Coulomb,
nonostante sia a simmetria sferica, perché decresce come $1/r$.
Sotto questa ipotesi, con riferimento alla
figura~\ref{fig:regioni-potenziale-sferico} possiamo suddividere lo spazio in
tre regioni: la regione di diffusione, in cui il potenziale di interazione è
sensibilmente diverso da zero, la regione intermedia in cui $U \ll l(l+1)/r^{2}$
e la zona di radiazione, in cui $kr \gg 1$ e quindi anche il termine di momento
angolare è trascurabile nell'hamiltoniana.  Nella zona di radiazione, cioè per
grandi valori di $r$, l'equazione di Schröedinger si riduce a
\begin{equation}
  \toder[2]{u_{kl}}{r^{2}} = -k^{2}u.
\end{equation}
La soluzione generale è data da
\begin{equation}
  u_{kl}(r) = D\e^{\uimm kr} + F\e^{-\uimm kr}.
\end{equation}
Il primo termine rappresenta un'onda sferica uscente, il secondo un'onda sferica
entrante.  Nel problema di diffusione è presente solo l'onda sferica uscente,
quindi $F = 0$ e
\begin{equation}
  R_{kl}(r) \sim \frac{\e^{\uimm kr}}{r}
\end{equation}
come avevo previsto nella forma asintotica~\eqref{eq:forma-asintotica}.

Nella regione intermedia l'equazione radiale è
\begin{equation}
  \toder[2]{u_{kl}}{r} - \frac{l(l+1)}{r^{2}}u_{kl} = -k^{2}u_{kl}.
\end{equation}
La soluzione di questa equazione è data dalla combinazione lineare delle
funzioni di Bessel sferiche $j_{l}$ e delle funzioni di Neumann sferiche $n_{l}$
\begin{equation}
  u_{kl}(r) = Brj_{l}(kr) + Crn_{l}(kr) \implies R_{kl}(r) = Bj_{l}(kr) +
  Cn_{l}(kr).
\end{equation}
Le funzioni sferiche di Bessel e di Neumann hanno i seguenti comportamenti
asintotici
\begin{subequations}
  \label{eq:asintoti-bessel}
  \begin{align}
    j_{l}(\rho) &\underset{\rho \to 0}{\sim} \frac{\rho^{l}}{(2l+1)!!}, \\
    \label{eq:jl-asintotico}
    j_{l}(\rho) &\underset{\rho \to \infty}{\sim} \frac{1}{\rho} \sin\bigg(\rho
    - l \frac{\pi}{2}\bigg), \\
    n_{l}(\rho) &\underset{\rho \to 0}{\sim} \frac{(2l-1)!!}{\rho^{l+1}}, \\
    n_{l}(\rho) &\underset{\rho \to \infty}{\sim} \frac{1}{\rho} -\cos\bigg(\rho
    - l \frac{\pi}{2}\bigg).
  \end{align}
\end{subequations}
quindi nell'origine le funzioni di Bessel convergono, mentre le funzioni di
Neumann divergono.  Per normalizzare $R_{kl}$ scegliamo i coefficienti $B$ e $C$
tali che $\abs{B}^{2} + \abs{C}^{2} = 1$, in particolare poniamo
$B = \cos\delta_{l}$ e $C = -\sin\delta_{l}$, quindi
\begin{equation}
  \label{eq:R-intermedio}
  R_{kl}(r) = j_{l}(kr)\cos\delta_{l} - n_{l}(kr)\sin\delta_{l}.
\end{equation}
L'equazione differenziale che soddisfa $R_{kl}$ è reale, quindi la soluzione può
essere scelta reale e anche i $\delta_{l}$ dovranno essere reali.  Con questa
posizione, usando le proprietà~\eqref{eq:asintoti-bessel} troviamo che il
comportamento asintotico per grandi valori di $kr$, quindi nella zona di
radiazione, di $R_{kl}$ è
\begin{equation}
  \label{eq:R-asintotico}
  R_{kl}(r) \underset{kr \to \infty}{\sim} \frac{\sin(kr - l\pi/2 +
    \delta_{l})}{kr}.
\end{equation}
Se non ci fosse potenziale di interazione, l'espressione~\eqref{eq:R-intermedio}
della funzione radiale sarebbe valida fino a $r=0$, non solo nella regione
intermedia.  Abbiamo osservato che le funzioni di Neumann sferiche nell'origine
divergono come $1/r^{l+1}$, ma la parte radiale della funzione d'onda non può
avere questo comportamento, quindi deve essere $\delta_{l} = 0$ per ogni $l$ e
$r$ in assenza di potenziale e in questo caso si avrebbe $R_{kl}(r) =
j_{l}(kr)$.
Questo risultato supporta lo sviluppo in onde parziali ipotizzato
nell'equazione~\eqref{eq:onde-parziali2}.  A questo punto possiamo confrontare
l'andamento asintotico~\eqref{eq:R-asintotico} di $R_{kl}(r)$ con
quello~\eqref{eq:jl-asintotico} del caso di potenziale nullo e riconosciamo che
l'effetto di un potenziale sferico a corto raggio a grandi distanze $r$ è quello
di introdurre uno sfasamento in ciascuna funzione radiale asintotica $R_{kl}(r)$
dello sviluppo in onde parziali~\eqref{eq:onde-parziali2}.

\subsection{Calcolo della sezione d'urto}
\label{sec:sez-urto-op}

Nel paragrafo~\ref{sec:sez-urto-mq} abbiamo visto che la sezione d'urto
differenziale è determinata dal comportamento asintotico della funzione d'onda.
I potenziali sferici a corto raggio introducono uno sfasamento negli stati
stazionari di diffusione, quindi ci aspettiamo che in questo caso la sezione
d'urto differenziale sia esprimibile in funzione degli sfasamenti.

Sostituiamo gli sviluppi in onde parziali~\eqref{eq:sviluppo-onda-piana} e
\eqref{eq:onde-parziali2} nell'espressione
asintotica~\eqref{eq:forma-asintotica} della funzione d'onda, con $A = 1$ per
semplicità, ricordando i comportamenti asintotici~\eqref{eq:jl-asintotico} e
\eqref{eq:R-asintotico}
\begin{multline}
  \label{eq:foo}
  \sum_{l}(2l+1)\uimm^{l}P_{l}(\cos\theta)A_{l}\frac{\sin(kr - l\pi/2 +
    \delta_{l})}{kr} \\
  = \sum_{l}(2l+1)\uimm^{l} P_{l}(\cos\theta)\frac{\sin(kr - l\pi/2)}{kr} +
  f(\theta,\phi)\frac{\e^{\uimm kr}}{r}.
\end{multline}
Usando la relazione $\sin x = (\e^{\uimm x} - \e^{-\uimm x})/(2\uimm)$ e
uguagliando ambo i membri così ottenuti i coefficienti di $\e^{-\uimm kr}$
troviamo
\begin{multline}
  \sum_{l}(2l+1)\uimm^{l}P_{l}(\cos\theta)A_{l}\exp(\uimm l\pi/2 -
  \uimm\delta_{l}) \\
  = \sum_{l}(2l+1)\uimm^{l}P_{l}(\cos\theta)\exp(\uimm l\pi/2).
\end{multline}
I polinomi di Legendre sono funzioni linearmente indipendenti, quindi affinché
l'equazione precedente sia valida devono essere uguali i coefficienti dei
$P_{l}$ dello stesso grado $l$, da cui ricaviamo che
\begin{equation}
  A_{l} = \e^{\uimm \delta_{l}}.
\end{equation}
Procedendo in maniera analoga, uguagliando i coefficienti di $\e^{\uimm kr}$
nella~\eqref{eq:foo} e ricordando il risultato appena determinato troviamo
l'ampiezza di diffusione
\begin{equation}
  \label{eq:ampiezza-diffusione-op}
  \begin{split}
    f(\theta,\phi) &= f(\theta) = \frac{1}{2\uimm k}\sum_{l}(2l+1)
    \overbrace{\uimm^{l}\e^{-\uimm l\pi/2}}^{\uimm^{l}(-\uimm)^{l}=1}
    (\e^{2\uimm\delta_{l}} - 1)P_{l}(\cos\theta) \\
    &= \frac{1}{2\uimm k}\sum_{l}(2l+1)(\e^{2\uimm\delta_{l}} - 1)
    P_{l}(\cos\theta) \\
    &= \frac{1}{k} \sum_{l} (2l+1) \sin\delta_{l} \e^{\uimm\delta_{l}}
    P_{l}(\cos\theta).
  \end{split}
\end{equation}
L'ampiezza di diffusione dipende solo dalla colatitudine $\theta$ perché, come
notato in precedenza, nelle diffusioni da campi sferici c'è simmetria
cilindrica.  Osserviamo inoltre che $f(\theta)$ non cambia per effetto di una
sostituzione $\delta_{l} \to \delta_{l} + \pi$.

La sezione d'urto differenziale è data dall'equazione~\eqref{eq:sezione-d'urto}
\begin{equation}
  \sigma(\theta,\phi) = \sigma(\theta) = \abs{f(\theta)}^{2}.
\end{equation}
La sezione d'urto totale si ricava integrando la sezione d'urto differenziale su
tutto l'angolo solido.  Ricordando la relazione di ortogonalità fra i polinomi
di Legendre
\begin{equation}
  \int_{-1}^{1}P_{l}(u)P_{l'}(u)\dd u = \frac{2\delta_{ll'}}{2l+1}
\end{equation}
ricaviamo
\begin{equation}
  \begin{split}
    \sigma_{\textup{tot}} &= \int \sigma(\theta) \sin\theta\dd\theta\dd\phi =
    \int_{0}^{2\pi}\dd \phi \int_{0}^{\pi} \abs{f(\theta)}^{2} \sin\theta
    \dd\theta \\
    &= 2\pi \frac{1}{k^{2}} \sum_{ll'}(2l+1)(2l'+1) \sin\delta_{l} \e^{-\uimm
      \delta_{l}} \sin\delta_{l'}\e^{\uimm \delta_{l'}}
    \int_{-1}^{1}P_{l}(\cos\theta)P_{l'}(\cos\theta) \dd(\cos\theta) \\
    &= \frac{4\pi}{k^{2}} \sum_{l}(2l+1)\sin^{2}\delta_{l}.
  \end{split}
\end{equation}
Utilizzando le altre espressioni dell'ampiezza di diffusione $f(\theta)$ si
possono trovare con calcoli analoghi espressioni differenti per la sezione
d'urto totale
\begin{equation}
  \sigma_{\textup{tot}} = \sum_{l} \sigma_{\textup{tot}}^{l} =
  \frac{\pi}{k^{2}} \sum_{l}(2l+1)\abs{1 - S_{l}}^{2} = \frac{2\pi}{k^{2}}
  \sum_{l}(2l+1)(1-\Re(S_{l})),
\end{equation}
con $S_{l} = \e^{2\uimm\delta_{l}}$.  Infine notiamo che ponendo $\theta = 0$
nell'ultimo membro dell'equazione~\eqref{eq:ampiezza-diffusione-op} e ricordando
che $P_{l}(1) = 1$ abbiamo
\begin{equation}
  f(0) = \frac{1}{k} \sum_{l}(2l+1)(\sin\delta_{l}\cos\delta_{l} + \uimm
  \sin^{2}\delta_{l}),
\end{equation}
quindi
\begin{equation}
  \sigma_{\textup{tot}} = \frac{4\pi}{k} \Im(f(0)).
\end{equation}
Questo risultato è noto come \emph{teorema ottico}.

Se il potenziale è non nullo per una distanza relativa $r < L$ ed è assente per
distanze maggiori è possibile trovare una formula che fornisce direttamente gli
sfasamenti $\delta_{l}$.  Possiamo considerare il punto $r = L$ come il
passaggio fra la regione di diffusione e la regione intermedia.  Bisogna
risolvere, eventualmente usando metodi numerici, l'equazione
differenziale~\eqref{eq:diff-R-op} nella regione di diffusione $r < L$ ed
effettuare il raccordo con la forma~\eqref{eq:R-intermedio} valutata nel punto
$r = L$.  Poiché la~\eqref{eq:diff-R-op} è un'equazione differenziale del
secondo ordine, ci saranno due soluzioni linearmente indipendenti, di cui solo
una non diverge per nell'origine, come succede per esempio in assenza di
potenziale con le funzioni sferiche di Bessel e Neumann.  Per determinare gli
sfasamenti possiamo imporre la continuità in $r=L$ di $R_{kl}$ e
$\ltoder{R_{kl}}{r}$, ma è più conveniente imporre che sia continua da sinistra in
$r=L$ la derivata logaritmica
\begin{equation}
  \gamma_{kl} = \toder{\log R_{kl}}{r} = \frac{1}{R_{kl}} \toder{R_{kl}}{r},
\end{equation}
che ha il vantaggio di essere indipendente dalla normalizzazione della funzione
radiale.  Sostituiamo l'espressione~\eqref{eq:R-intermedio} di $R_{kl}$, valida
nella regione intermedia, nell'equazione precedente e valutiamo il risultato nel
punto $r = L$
\begin{equation}
  \gamma_{kl} = \frac{k(j'_{l}(kL)\cos\delta_{l} -
    n'_{l}(kL)\sin\delta_{l})}{j_{l}(kL)\cos\delta_{l} -
    n_{l}(kL)\sin\delta_{l}},
\end{equation}
in cui $j'(\rho) = \partial_{\rho}j(\rho)$.  Riarrangiando l'equazione
precedente è possibile ricavare la tangente di $\delta_{l}$
\begin{equation}
  \label{eq:tangente-sfasamento}
  \tan\delta_{l} = \frac{kj'_{l}(kL) - \gamma_{kl}j_{l}(kL)}{kn'_{l}(kL) -
    \gamma_{kl}n_{l}(kL)}.
\end{equation}
Si definiscono le funzioni di Hankel sferiche
\begin{subequations}
  \begin{align}
    h_{l}^{(1)}(\rho) &= j_{l}(\rho) + \uimm n_{l}(\rho), \\
    h_{l}^{(2)}(\rho) &= j_{l}(\rho) - \uimm n_{l}(\rho)
  \end{align}
\end{subequations}
e vediamo come possono essere utilizzate per calcolare gli sfasamenti.  Tramite
gli sviluppi in esponenziali delle funzioni seno e coseno si trova la seguente
relazione valida in generale
\begin{equation}
  \e^{2\uimm x} = \frac{\uimm \tan x + 1}{1 - \uimm \tan x},
\end{equation}
da cui
\begin{equation}
  S_{l} = \e^{2\uimm\delta_{l}} = -\frac{R_{kl}(L)k h_{l}^{(2)}{}'(kL) -
    R'_{kl}(L)h_{l}^{(2)}(kL)}{R_{kl}(L)k h_{l}^{(1)}{}'(kL) -
    R'_{kl}(L)h_{l}^{(1)}(kL)} = -\left.\frac{W[R_{kl}(r),
      h_{l}^{(2)}(kr)]}{W[R_{kl}(r), h_{l}^{(1)}(kr)]}\right|_{r = L}.
\end{equation}
Il simbolo $W[\cdot,\cdot]$ indica il wronskiano delle due funzioni, con
derivate rispetto alla coordinata $r$, quindi per le funzioni di Hankel si ha
$\partial_{r}h(kr) = k \partial_{\rho}h(\rho) = k h'(\rho)$.

Se il potenziale non è esattamente nullo per $r > L$ ma è comunque a corto
raggio, possiamo calcolare gli sfasamenti usando ancora la
formula~\eqref{eq:tangente-sfasamento} nel limite $L \to \infty$, ricordando che
anche $\gamma_{kl}$ dipende da $L$.  Infine si può dimostrare che
\begin{equation}
  \delta_{l} \underset{l \to \infty}{\sim} \frac{1}{l!}.
\end{equation}


\section{Approssimazione di Born}
\label{sec:approx-born}

\section{Operatori di diffusione}
\label{sec:operatori-diffusione}

\phantomsection
\addcontentsline{toc}{section}{\refname}
\nocite{*}
\printbibliography

\end{document}

%%% Local Variables:
%%% mode: latex
%%% TeX-master: t
%%% End:
