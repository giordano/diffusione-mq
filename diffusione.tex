\documentclass[a4paper,fleqn,twoside]{article}
\usepackage[T1]{fontenc}
\usepackage[light,slantedGreeks]{kpfonts}
\usepackage[utf8]{inputenc}
\usepackage[italian]{babel}

%%%%% Pacchetti caricati
\usepackage{layaureo}
\usepackage[autostyle=true]{csquotes}
\usepackage[style=numeric,hyperref,abbreviate=false,backend=biber]{biblatex}
% `mathtools' serve per definire la norma e il valore assoluto
\usepackage{mathtools}
% `bm' serve per scrivere i vettori in corsivo con il comando
% `\bm{vettore}'. Deve sostituire il comando `\mathbf{vettore}' perché questo
% restituisce erroneamente lettere in tondo, non in corsivo e non funziona con
% le lettere greche.
\usepackage{bm}
\usepackage{xcolor}
\usepackage{hyperref}

% impostazioni per il pacchetto `hyperref'. Per l'elenco di tutte le opzioni del
% documento consulta il manuale: `texdoc hyperref'
\hypersetup{
  pdftitle={Introduzione alla teoria della diffusione in meccanica quantistica},
  pdfauthor={Mosè Giordano},
  breaklinks=true,% permette di spezzare i link su più righe
  bookmarksnumbered,% inserisce i numeri delle sezioni nei segnalibri
  hidelinks % link neri e senza bordi colorati, adatto per la stampa
}

\addbibresource{bibliografia.bib} % nome del file contenente la bibliografia

%%%%% Comandi personalizzati
% ridefinisco i comandi per alcune lettere greche in modo che si usino le
% varianti
\renewcommand{\phi}{\varphi}
\renewcommand{\epsilon}{\varepsilon}

% comando per evidenziare le sezioni da completare o espandere.  NOTA: è un
% comando provvisorio, serve solo durante la scrittura del testo, ricordarsi di
% eliminarlo quando non serve più.
\newcommand{\completare}[1]{\textcolor{red}{#1}}

% Operatori
\newcommand*{\dd}{\mathop{}\!\mathrm{d}} % Operatore differenziale \dd
\DeclareMathOperator{\e}{\mathrm{e}} % Numero di Eulero
\DeclareMathOperator{\uimm}{\mathrm{i}} % unità immaginaria
% Operatore valore assoluto \abs{x}. Usa \abs*{} per le frazioni
\DeclarePairedDelimiter{\abs}{\lvert}{\rvert}
% Operatore norma \norm{x}. Usa \norm*{} per le frazioni
\DeclarePairedDelimiter{\norm}{\lVert}{\rVert}

%% Derivate
% Derivata totale: \toder[ordine]{funzione}{variabile}
\newcommand*{\toder}[3][]{\frac{{\dd^{#1}}#2}{\dd {#3}^{#1}}}
% Derivata parziale \parder[ordine]{funzione}{variabile}
% Per la definizione del comando `parder' (per inserire le derivate parziali)
% vedi
% http://www.guitex.org/home/index.php?option=com_kunena&func=view&catid=5&id=42178&Itemid=60#42199
\makeatletter
\newcommand{\parder}[2]{\begingroup
  \@tempswafalse\toks@={}\count@=\z@
  \@for\next:=#2\do
    {\expandafter\check@var\next\@nil
     \advance\count@\parder@exp
     \if@tempswa
       \toks@=\expandafter{\the\toks@\,}%
     \else
       \@tempswatrue
     \fi
     \toks@=\expandafter{\the\expandafter\toks@\expandafter\partial\parder@var}}%
  \frac{\partial\ifnum\count@=\@ne\else^{\number\count@}\fi#1}{\the\toks@}%
  \endgroup}
\def\check@var{\@ifstar{\mult@var}{\one@var}}
\def\mult@var#1#2\@nil{\def\parder@var{#2^{#1}}\def\parder@exp{#1}}
\def\one@var#1\@nil{\def\parder@var{#1}\chardef\parder@exp\@ne}
\makeatother

% Derivate per le formule in linea (usare \frac in linea è eccessivo). La `l'
% iniziale nel nome distingue questi comandi da quelli per le formule fuori
% corpo. Non uso `\dd' ma `\mathrm{d}' perché nelle formule in linea `\dd'
% aggiunge una spaziatura non adatta. Non sono dei comandi bellissimi, ma
% permettono di passare facilmente da formula in linea a fuori corpo e viceversa
% cambiando una lettera.
% Derivata totale: \ltoder[ordine]{funzione}{variabile}
\newcommand*{\ltoder}[3][]{\mathrm{d}^{#1}#2 / \mathrm{d} {#3}^{#1}}
% Derivata parziale: \lparder[ordine]{funzione}{variabile}
\newcommand*{\lparder}[3][]{\partial^{#1} #2 / \partial {#3}^{#1}}
% NOTA: `\parder' e `\lparder' non sono completamente interscambiabili, il primo
% comando è molto più complesso e permette di inserire le derivate miste, a
% differenza del secondo.

% Versore. Esempi: versore x: `\versore{x}', versore i: \versor{\imath}, versore
% j: \versor{\jmath} (solo `i' e `j' richiedono `\imath' e `\jmath', altrimenti
% il puntino litiga con `\hat')
\newcommand*{\versor}[1]{\hat{\bm{#1}}}


\title{Introduzione alla teoria della diffusione \\ in meccanica quantistica}
\author{Mosè Giordano}

\begin{document}
\maketitle
\tableofcontents

\section{\completare{Concetti generali}}
\label{sec:concetti-generali}

Flusso di proiettili per unità di area e tempo
\begin{equation}
  J_{a} = \frac{\text{numero proiettili}}{\text{area}\cdot\text{tempo}} =
  \frac{n_{a}}{S_{\textup{t}} T}
\end{equation}
$S_{\textup{t}} = \text{sezione trasversale}$, $T = \text{tempo}$.  Numero di
processi di diffusione per unità di tempo
\begin{equation}
  n_{t} = \sigma_{\textup{tot}} J_{a} N_{a}
\end{equation}
$N_{A} = \text{numero di bersagli $A$}$,
$\sigma_{\textup{tot}} = \text{\emph{sezione d'urto totale}}$.

Numero $\dd n_{b}$ di particelle diffuse per unità di tempo nell'angolo solido
$\dd\Omega_{b}$.  È intuitivo capire che $\dd n_{b}$ deve essere proporzionale
al flusso $J_{a}$ di proiettili e all'angolo solido $\dd\Omega_{b}$ in cui si
vanno a rilevare le particelle diffuse.  Indichiamo con $\sigma(\theta,\phi)$ il
coefficiente di proporzionalità
\begin{equation}
  \dd n_{b} = \frac{\text{numero di eiettili in $\dd\Omega_{b}$}}{T} = J_{a}
  \dd\Omega_{b} \sigma(\theta,\phi).
\end{equation}
La quantità $\sigma(\theta,\phi)$ prende il nome di
\emph{sezione d'urto differenziale}.  Dividendo $\dd n_{b}$ per l'area
$S_{\textup{d}}$ del rivelatore otteniamo il flusso $J_{b}$ di particelle
diffuse per unità di tempo e di area.  L'area $S_{\textup{d}}$ del rivelatore è
data dal prodotto fra il quadrato della distanza $r$ in cui si trova il
rivelatore e l'apertura angolare $\dd\Omega_{b}$ del rivelatore:
$S_{\textup{d}} = r^{2}\dd\Omega_{b}$.  Allora
\begin{equation}
  J_{b} = \frac{\dd n_{b}}{S_{\textup{d}}} = \frac{J_{a} \sigma(\theta,\phi)
    \dd\Omega_{b}}{r^{2} \dd\Omega_{b}} = \frac{J_{a}
    \sigma(\theta,\phi)}{r^{2}},
\end{equation}
da cui ricaviamo che la sezione d'urto differenziale è data da
\begin{equation}
  \label{eq:sezione-d'urto}
  \sigma(\theta,\phi) = \frac{J_{b}r^{2}}{J_{a}}.
\end{equation}
La sezione d'urto differenziale dipende anche dall'energia del fascio di
proiettili e dallo specifico canale di diffusione seguito.  Nota
$\sigma(\theta,\phi)$, la sezione d'urto totale si ottiene integrando su tutto
l'angolo solido
\begin{equation}
  \sigma_{\textup{tot}} = \int \sigma(\theta,\phi) \dd\Omega = \int_{0}^{2\pi}
  \dd \phi \int_{0}^{\pi} \sin \theta \dd\theta \sigma(\theta,\phi).
\end{equation}

\subsection{Sistemi di riferimento del laboratorio e del centro di massa}
\label{sec:sistemi-riferimento}

\section{Diffusione in meccanica quantistica}
\label{sec:meccanica-quantistica}

Tutto quello che abbiamo detto finora è valido in generale.  Svilupperemo ora la
teoria della diffusione nell'ambito della meccanica quantistica non relativistica.

L'hamiltoniana per un sistema di due particelle senza spin interagenti fra loro
è
\begin{equation}
  H = -\frac{\hslash^{2}}{2m_{1}} \nabla_{1}^{2}
  -\frac{\hslash^{2}}{2m_{2}}\nabla_{2}^{2} + V(\bm{r_{1}} - \bm{r}_{2}).
\end{equation}
Stiamo facendo l'ipotesi che le due particelle interagiscano con un potenziale
$V$ che dipende solo dalla posizione relativa $\bm{r_{1}} - \bm{r}_{2}$ dei due
corpi.  Data la forma dell'hamiltoniana è naturale trattare il seguente problema
come il solito problema dei due corpi, quindi definiamo la posizione $\bm{R}$
del centro di massa e la posizione relativa $\bm{r}$ rispettivamente come
\begin{subequations}
  \begin{align}
    \bm{R} &= \frac{m_{1}\bm{r}_{1} + m_{2}\bm{r}_{2}}{m_{1} + m_{2}}, \\
    \bm{r} &= \bm{r_{1}} - \bm{r}_{2}.
  \end{align}
\end{subequations}
In questo modo l'hamiltoniana assume la forma più semplice
\begin{equation}
  \label{eq:hamiltoniana}
  H = -\frac{\hslash^{2}}{2(m_{1} + m_{2})}\nabla_{\bm{R}}^{2} -
  \frac{\hslash^{2}}{2m}\nabla_{\bm{r}}^{2} + V(\bm{r}).
\end{equation}
Il primo termine è l'energia cinetica del centro di massa, gli ultimi due sono
l'energia della particella fittizia di massa ridotta
$m = m_{1}m_{2}/(m_{1}+m_{2})$.  Come noto, possiamo cercare autostati
dell'hamiltoniana~\eqref{eq:hamiltoniana} della forma separabile
$\Psi(\bm{R}, \bm{r}) = \Phi(\bm{R})\psi(\bm{r})$.  In particolare,
$\psi(\bm{r})$ deve soddisfare l'equazione
\begin{equation}
  \bigg(-\frac{\hslash^{2}}{2m}\nabla^{2} + V(\bm{r})\bigg)\psi(\bm{r}) =
  E\psi(\bm{r}).
\end{equation}
Il moto del centro di massa non è di nostro interesse poiché la sua hamiltoniana
è semplicemente quella di particella libera, dunque ci occuperemo di studiare
solo il moto relativo fra la particella incidente e il bersaglio.

La forma della funzione d'onda $\psi(\bm{r})$ deve essere determinata dalle
condizioni al contorno.  Nel paragrafo~\ref{sec:concetti-generali} abbiamo visto
che nei problemi di diffusione è presente un flusso $J_{a}$ incidente sul
bersaglio e un flusso $J_{b}$ diffuso dal bersaglio e che si allontana in tutte
le direzioni.  Ci aspettiamo dunque che la funzione d'onda $\psi(\bm{r})$ sia la
somma di una funzione d'onda incidente $\psi_{a}$, che rappresenta il flusso di
proiettili, e una funzione d'onda di diffusione $\psi_{b}$, che rappresenta gli
eiettili
\begin{equation}
  \psi(\bm{r}) = \psi_{a}(\bm{r}) + \psi_{b}(\bm{r}).
\end{equation}
Facendo un'analogia con la meccanica ondulatoria classica possiamo prevedere una
forma più esplicita delle due componenti.  Possiamo considerare $\psi_{a}$ come
un'onda a grande distanza dell'origine, approssimabile localmente con un'onda
piana del tipo $\e^{\uimm \bm{k}\cdot\bm{r}}$, che urta contro un ostacolo, cioè
il bersaglio nel problema di diffusione, e si diffonde in tutto lo spazio come
un'onda sferica uscente del tipo $f(\theta,\phi)\e^{\uimm kr}/r$, con
$f(\theta,\phi)$ ampiezza dell'onda di diffusione.  Quindi ci aspettiamo che la
funzione d'onda $\psi(\bm{r})$ sia del tipo
\begin{equation}
  \label{eq:forma-onda}
  \psi(\bm{r}) = \psi_{a}(\bm{r}) + \psi_{b}(\bm{r}) =A \bigg( \e^{\uimm
    \bm{k}\cdot\bm{r}} + f(\theta,\phi)\frac{\e^{\uimm kr}}{r} \bigg),
\end{equation}
con $A$ fattore di normalizzazione e $\bm{k}$ vettore d'onda, legato all'energia
$E$ delle particelle incidenti, di massa $m$, dalla nota relazione
\begin{equation}
  E = \frac{\hslash^{2}k^{2}}{2m}.
\end{equation}
Nei prossimi paragrafi vedremo che la forma~\eqref{eq:forma-onda} della funzione
d'onda in un problema di diffusione, ricavata qui in base a un'analogia
classica, è ben giustificata sotto opportune ipotesi.

Lo stato quantico $\psi(\bm{r})$ di una particella non descrive esattamente la
sua posizione ma l'ampiezza della probabilità di trovare la particella nella
posizione $\bm{r}$.  Analogamente, il flusso che dobbiamo considerare per
calcolare la sezione d'urto differenziale è il flusso di probabilità, cioè la
probabilità per unità di tempo che la particella attraversi l'area unitaria.  È
noto che per una particella di massa $m$ nello stato $\psi$, il flusso di
probabilità vale
\begin{equation}
  \bm{J} = \frac{\hslash}{m} \Im(\psi^{*}\nabla\psi).
\end{equation}
Applicando questa equazione a ciascuna componente della funzione
d'onda~\eqref{eq:forma-onda} abbiamo i seguenti valori per i flussi
\begin{subequations}
  \begin{align}
    \bm{J}_{a} &= \frac{\hslash}{m}\Im(\psi_{a}^{*}\nabla\psi_{a}) =
    \frac{\hslash}{m}\Im (A^{*}\e^{-\uimm \bm{k}\cdot\bm{r}} A\uimm
    \bm{k}\e^{\uimm \bm{k}\cdot\bm{r}}) = \frac{\abs{A}^{2}\hslash \bm{k}}{m}, \\
    \begin{split}
      (\bm{J}_{b})_{\textup{r}} &=
      \frac{\hslash}{m}\Im\bigg(\psi_{b}^{*}\parder{\psi_{b}}{r}\bigg) =
      \frac{\hslash}{m}\Im \bigg((Af(\theta,\phi))^{*} \frac{\e^{-\uimm kr}}{r}
      Af(\theta,\phi)\frac{\e^{\uimm kr}}{r}\bigg(\uimm k - \frac{1}{r}\bigg)
      \bigg) \\
      &= \abs{Af(\theta,\phi)}^{2} \frac{\hslash k}{mr^{2}}.
    \end{split}
  \end{align}
\end{subequations}
$(\bm{J}_{b})_{\textup{r}}$ è la componente radiale del flusso della componente
di diffusione della funzione d'onda.  Inserendo questi risultati
nell'equazione~\eqref{eq:sezione-d'urto} troviamo che la sezione d'urto
differenziale è
\begin{equation}
  \sigma(\theta,\phi) = \frac{J_{b}r^{2}}{J_{a}} = \abs{f(\theta,\phi)}^{2}.
\end{equation}
La sezione d'urto differenziale è la quantità di maggior interesse nei problemi
di diffusione perché è quella misurabile sperimentalmente e abbiamo visto che è
uguale al modulo quadro di $f(\theta,\phi)$, chiamata per questo motivo
\emph{ampiezza di diffusione}.  Nei due prossimi paragrafi studieremo due metodi
alternativi per calcolare l'ampiezza di diffusione: lo sviluppo in onde parziali
e l'approssimazione di Born.

Abbiamo trascurato i gradi di libertà interni delle due particelle, dunque
stiamo implicitamente limitando la nostra attenzione a problemi di diffusione
elastica.

\section{Sviluppo in onde parziali}
\label{sec:onde-parziali}

Il primo metodo che studieremo è particolarmente utile nei casi in cui il
potenziale di interazione ha simmetria sferica, cioè dipende solo dal modulo $r$
della distanza relativa fra i due corpi: $V(\bm{r}) = V(r)$.  L'equazione di
Schröedinger assume la forma
\begin{equation}
  (\nabla^{2} + k^{2} - U(r)) \psi(\bm{r}) = 0,
\end{equation}
con $U(r) = (2m/\hslash^{2})V(r)$.  In questo caso sappiamo che una soluzione
dell'equazione precedente è del tipo separabile
$R_{kl}(r)Y_{l}^{m}(\theta,\phi)$, in cui $Y_{l}^{m}(\theta,\phi))$ è
un'armonica sferica e $R_{kl}(r)$ è una funzione puramente radiale che può
essere posta nella forma $R_{kl}(r) = u_{kl}(r)/r$.  La soluzione generale
dell'equazione precedente sarà una combinazione lineare delle soluzioni appena
illustrate
\begin{equation}
  \psi(\bm{r}) = \sum_{l = 0}^{+\infty} \sum_{m = -l}^{l} a_{lm}
  \frac{u_{kl}(r)}{r}Y_{l}^{m}(\theta,\phi).
\end{equation}
L'equazione radiale che soddisfa $u_{kl}$ è
\begin{equation}
  \bigg(\toder[2]{}{r} + k^{2} - U(r) - \frac{l(l+1)}{r^{2}}\bigg)u_{kl}(r) = 0.
\end{equation}

% TODO: fare la figura dello spazio diviso in tre parti
Supponiamo che il potenziale di interazione sia a \emph{rapida decrescenza}, o
\emph{a corto raggio}, vale a dire per $r$ tendente all'infinito va a $0$ più
rapidamente di $1/r^{2}$
\begin{equation}
  \lim_{r \to \infty} r^{2}V(r) \in \mathbb{R}.
\end{equation}
In questo modo stiamo escludendo dalla trattazione il potenziale di Coulomb,
nonostante sia a simmetria sferica, perché decresce come $1/r$.  Sotto questa
ipotesi possiamo suddividere lo spazio in tre regioni: la regione di diffusione,
in cui il potenziale di interazione è diverso da zero, la regione intermedia in
cui $U \ll l(l+1)/r^{2}$ e la zona di radiazione, in cui $kr \gg 1$ e quindi
anche il termine di momento angolare è trascurabile nell'hamiltoniana.  Nella
zona di radiazione, cioè per grandi valori di $r$, l'equazione di Schröedinger
si riduce a
\begin{equation}
  \toder[2]{u_{kl}}{r^{2}} = -k^{2}u.
\end{equation}
La soluzione generale è data da
\begin{equation}
  u_{kl}(r) = D\e^{\uimm kr} + F\e^{-\uimm kr}.
\end{equation}
Il primo termine rappresenta un'onda sferica uscente, il secondo un'onda sferica
entrante.  Nel problema di diffusione è presente solo l'onda sferica uscente,
quindi $F = 0$ e
\begin{equation}
  R_{kl}(r) \sim \frac{\e^{\uimm kr}}{r}.
\end{equation}


\section{Approssimazione di Born}
\label{sec:approx-born}

\section{Operatori di diffusione}
\label{sec:operatori-diffusione}

\phantomsection
\addcontentsline{toc}{section}{\refname}
\nocite{*}
\printbibliography

\end{document}

%%% Local Variables:
%%% mode: latex
%%% TeX-master: t
%%% End:
