\documentclass[a4paper,fleqn,twoside,12pt]{article}
\usepackage[T1]{fontenc}
\usepackage[light,slantedGreeks]{kpfonts}
\usepackage[utf8]{inputenc}
\usepackage[italian]{babel}

%%%%% Pacchetti caricati
\usepackage{layaureo}
\usepackage[autostyle=true]{csquotes}
\usepackage[style=authoryear,hyperref,backend=biber]{biblatex}
\usepackage[font=small,format=hang,labelfont=bf]{caption}
\usepackage{sidecap,subfig}
\usepackage{braket}
\usepackage{tikz}
\usepackage{mathtools} % per definire il valore assoluto
\usepackage{bm} % per i vettori: \bm{x}
\usepackage{hyperref}

\usetikzlibrary{calc,decorations.markings}

% impostazioni per il pacchetto `hyperref'. Per l'elenco di tutte le opzioni del
% documento consulta il manuale: `texdoc hyperref'
\hypersetup{
  pdftitle={Introduzione alla teoria della diffusione in meccanica quantistica},
  pdfauthor={Mosè Giordano},
  breaklinks=true,% permette di spezzare i link su più righe
  bookmarksnumbered,% inserisce i numeri delle sezioni nei segnalibri
  hidelinks, % link neri e senza bordi colorati, adatto per la stampa
  pdfkeywords={diffusione, meccanica quantistica, scattering, quantum
    mechanics},
}

\addbibresource{bibliografia.bib} % nome del file contenente la bibliografia

%%%%% Comandi personalizzati
% ridefinisco i comandi per alcune lettere greche in modo che si usino le
% varianti
\renewcommand{\phi}{\varphi}
\renewcommand{\epsilon}{\varepsilon}

% comando per evidenziare le sezioni da completare o espandere.  NOTA: è un
% comando provvisorio, serve solo durante la scrittura del testo, ricordarsi di
% eliminarlo quando non serve più.
\newcommand{\completare}[1]{\textcolor{red}{#1}}

% Operatori
\newcommand*{\dd}{\mathop{}\!\mathrm{d}} % Operatore differenziale \dd
\DeclareMathOperator{\e}{\mathrm{e}} % Numero di Eulero
\DeclareMathOperator{\uimm}{\mathrm{i}} % unità immaginaria
% Operatore valore assoluto \abs{x}. Usa \abs*{} per le frazioni
\DeclarePairedDelimiter{\abs}{\lvert}{\rvert}

%% Derivate
% Derivata totale: \toder[ordine]{funzione}{variabile}
\newcommand*{\toder}[3][]{\frac{{\dd^{#1}}#2}{\dd {#3}^{#1}}}
% Derivata parziale \parder[ordine]{funzione}{variabile}
% Per la definizione del comando `parder' (per inserire le derivate parziali)
% vedi
% http://www.guitex.org/home/index.php?option=com_kunena&func=view&catid=5&id=42178&Itemid=60#42199
\makeatletter
\newcommand{\parder}[2]{\begingroup
  \@tempswafalse\toks@={}\count@=\z@
  \@for\next:=#2\do
    {\expandafter\check@var\next\@nil
     \advance\count@\parder@exp
     \if@tempswa
       \toks@=\expandafter{\the\toks@\,}%
     \else
       \@tempswatrue
     \fi
     \toks@=\expandafter{\the\expandafter\toks@\expandafter\partial\parder@var}}%
  \frac{\partial\ifnum\count@=\@ne\else^{\number\count@}\fi#1}{\the\toks@}%
  \endgroup}
\def\check@var{\@ifstar{\mult@var}{\one@var}}
\def\mult@var#1#2\@nil{\def\parder@var{#2^{#1}}\def\parder@exp{#1}}
\def\one@var#1\@nil{\def\parder@var{#1}\chardef\parder@exp\@ne}
\makeatother

% Derivate per le formule in linea (usare \frac in linea è eccessivo). La `l'
% iniziale nel nome distingue questi comandi da quelli per le formule fuori
% corpo. Non uso `\dd' ma `\mathrm{d}' perché nelle formule in linea `\dd'
% aggiunge una spaziatura non adatta. Non sono dei comandi bellissimi, ma
% permettono di passare facilmente da formula in linea a fuori corpo e viceversa
% cambiando una lettera.
% Derivata totale: \ltoder[ordine]{funzione}{variabile}
\newcommand*{\ltoder}[3][]{\mathrm{d}^{#1}#2 / \mathrm{d} {#3}^{#1}}
% Derivata parziale: \lparder[ordine]{funzione}{variabile}
\newcommand*{\lparder}[3][]{\partial^{#1} #2 / \partial {#3}^{#1}}
% NOTA: `\parder' e `\lparder' non sono completamente interscambiabili, il primo
% comando è molto più complesso e permette di inserire le derivate miste, a
% differenza del secondo.

% Versore. Esempi: versore x: `\versore{x}', versore i: \versor{\imath}, versore
% j: \versor{\jmath} (solo `i' e `j' richiedono `\imath' e `\jmath', altrimenti
% il puntino litiga con `\hat')
\newcommand*{\versor}[1]{\hat{\bm{#1}}}

% Il comando `\allinea' deve essere dato come valore dell'opzione `baseline' del
% comando `\tikz' per allineare verticalmente i disegni in linea.  Vedi
% http://tex.stackexchange.com/questions/59658/use-of-tikzpicture-matrix-in-align-or-gather-environment/59660#comment126261_59660
\newcommand\allinea{-\the\dimexpr\fontdimen22\textfont2\relax}

\tikzset{
  % vedi http://tex.stackexchange.com/a/39283
  middlearrow/.style={decoration={markings,mark= at position 0.5 with
      {\arrow{#1}},},postaction={decorate}},
  % vedi http://www.texample.net/tikz/examples/pgf-version-2/
  dot/.style={fill=black,circle,minimum size=2.5pt,inner sep=0}
}


\title{Introduzione alla teoria della diffusione \\ in meccanica quantistica}
\author{Mosè Giordano}

\begin{document}
\maketitle
{\small\tableofcontents}

\phantomsection
\addcontentsline{toc}{section}{Sommario}
\section*{Sommario}
\label{sec:sommario}

Queste note presentano un'introduzione alla teoria della diffusione da
potenziale nella meccanica quantistica non relativistica, senza alcuna pretesa
di completezza.  Sono una rielaborazione personale degli appunti delle lezioni
di Fisica Teorica tenute dal professor Luigi Martina nel corso di Laurea
Magistrale in Fisica all'Università del Salento nell'anno accademico 2011-2012.
I testi di riferimento seguiti
sono~\textcites{ballentine:quantum-mechanics,cohen:quantum-mechanics}, in
aggiunta ho integrato del materiale preso
da~\textcites{gottfried:quantum-mechanics,griffiths:introduction-qm,
  landau:meccanica-quantistica}.
Ogni errore presente in queste note è da attribuire a me.

\section{\completare{Concetti generali}}
\label{sec:concetti-generali}

Flusso di proiettili per unità di area e tempo
\begin{equation}
  J_{a} = \frac{\text{numero proiettili}}{\text{area}\cdot\text{tempo}} =
  \frac{n_{a}}{S_{\textup{t}} T}
\end{equation}
$S_{\textup{t}} = \text{sezione trasversale}$, $T = \text{tempo}$.  Numero di
processi di diffusione per unità di tempo
\begin{equation}
  n_{t} = \sigma_{\textup{tot}} J_{a} N_{a}
\end{equation}
$N_{A} = \text{numero di bersagli $A$}$,
$\sigma_{\textup{tot}} = \text{\emph{sezione d'urto totale}}$.

Numero $\dd n_{b}$ di particelle diffuse per unità di tempo nell'angolo solido
$\dd\Omega_{b}$.  È intuitivo capire che $\dd n_{b}$ deve essere proporzionale
al flusso $J_{a}$ di proiettili e all'angolo solido $\dd\Omega_{b}$ in cui si
vanno a rilevare le particelle diffuse.  Indichiamo con $\sigma(\theta,\phi)$ il
coefficiente di proporzionalità
\begin{equation}
  \dd n_{b} = \frac{\text{numero di eiettili in $\dd\Omega_{b}$}}{T} = J_{a}
  \dd\Omega_{b} \sigma(\theta,\phi).
\end{equation}
La quantità $\sigma(\theta,\phi)$ prende il nome di
\emph{sezione d'urto differenziale}.  Dividendo $\dd n_{b}$ per l'area
$S_{\textup{d}}$ del rivelatore otteniamo il flusso $J_{b}$ di particelle
diffuse per unità di tempo e di area.  L'area $S_{\textup{d}}$ del rivelatore è
data dal prodotto fra il quadrato della distanza $r$ in cui si trova il
rivelatore e l'apertura angolare $\dd\Omega_{b}$ del rivelatore:
$S_{\textup{d}} = r^{2}\dd\Omega_{b}$.  Allora
\begin{equation}
  J_{b} = \frac{\dd n_{b}}{S_{\textup{d}}} = \frac{J_{a} \sigma(\theta,\phi)
    \dd\Omega_{b}}{r^{2} \dd\Omega_{b}} = \frac{J_{a}
    \sigma(\theta,\phi)}{r^{2}},
\end{equation}
da cui ricaviamo che la sezione d'urto differenziale è data da
\begin{equation}
  \label{eq:sez-urto}
  \sigma(\theta,\phi) = \frac{J_{b}r^{2}}{J_{a}}.
\end{equation}
La sezione d'urto differenziale dipende anche dall'energia del fascio di
proiettili e dallo specifico canale di diffusione seguito.  Nota
$\sigma(\theta,\phi)$, la sezione d'urto totale si ottiene integrando su tutto
l'angolo solido
\begin{equation}
  \sigma_{\textup{tot}} = \int \sigma(\theta,\phi) \dd\Omega = \int_{0}^{2\pi}
  \dd \phi \int_{0}^{\pi} \sin \theta \dd\theta \sigma(\theta,\phi).
\end{equation}

% % TODO: scrivere questo paragrafo
% \subsection{Sistemi di riferimento del laboratorio e del centro di massa}
% \label{sec:sistemi-riferimento}

\section{Diffusione in meccanica quantistica}
\label{sec:meccanica-quantistica}

Tutto quello che abbiamo detto finora è valido in generale.  Introdurremo ora la
teoria della diffusione da potenziale nell'ambito della meccanica quantistica
non relativistica.  Prima di procedere precisiamo le ipotesi sotto le quali
affronteremo il problema
\begin{itemize}
\item supponiamo che le particelle coinvolte nella diffusione siano prive di
  spin;
\item non consideriamo la struttura interna delle particelle.  In questo modo
  escludiamo dalla trattazione le diffusioni anelastiche e ci occuperemo solo di
  quelle elastiche;
\item supponiamo che il bersaglio sia sufficientemente piccolo da poter
  trascurare processi di diffusione multipla;
\item trascuriamo la possibilità di coerenza fra onde diffuse dalle differenti
  particelle che costituiscono il bersaglio;
\item supponiamo che l'interazione fra proiettile e bersaglio sia descritta da
  un potenziale $V$ dipendente dalla posizione relativa fra le due particelle
  $\bm{r}_{1} - \bm{r}_{2}$: $V = V(\bm{r}_{1} - \bm{r}_{2})$.  Questo ci
  permetterà di adottare il formalismo noto del problema dei due corpi.
\end{itemize}
Abbiamo fatto queste assunzioni non perché non siano possibili casi differenti
(sono per esempio molto importanti i casi di diffusione di particelle dotate di
spin e di diffusioni anelastiche) ma solo per semplificare la presente
trattazione.

L'hamiltoniana per un sistema di due particelle senza spin interagenti fra loro
è
\begin{equation}
  H = -\frac{\hslash^{2}}{2m_{1}} \nabla_{1}^{2}
  -\frac{\hslash^{2}}{2m_{2}}\nabla_{2}^{2} + V(\bm{r}_{1} - \bm{r}_{2}).
\end{equation}
Come anticipato, tratteremo il problema come il solito problema dei due corpi,
quindi definiamo la posizione $\bm{R}$ del centro di massa e la posizione
relativa $\bm{r}$ rispettivamente come
\begin{subequations}
  \begin{align}
    \bm{R} &= \frac{m_{1}\bm{r}_{1} + m_{2}\bm{r}_{2}}{m_{1} + m_{2}}, \\
    \bm{r} &= \bm{r}_{1} - \bm{r}_{2}.
  \end{align}
\end{subequations}
In questo modo l'hamiltoniana assume la forma più semplice
\begin{equation}
  \label{eq:hamiltoniana}
  H = -\frac{\hslash^{2}}{2(m_{1} + m_{2})}\nabla_{\bm{R}}^{2} -
  \frac{\hslash^{2}}{2m}\nabla^{2} + V(\bm{r}).
\end{equation}
Il primo termine è l'energia cinetica del centro di massa, gli ultimi due sono
l'energia della particella fittizia di massa ridotta
$m = m_{1}m_{2}/(m_{1}+m_{2})$.  Vogliamo determinare gli autovalori
dell'hamiltoniana e poiché essa è indipendente dal tempo possiamo considerare
gli autostati stazionari.  In particolare siamo interessati agli autostati con
valori positivi dell'energia $E = \hslash^{2}k^{2}/2m$, associati a stati non
legati, perché se lo stato delle due particelle fosse legato sarebbe nulla la
probabilità di trovarle a distanza reciproca infinita.  Grazie alla forma
dell'hamiltoniana~\eqref{eq:hamiltoniana}, possiamo cercare suoi autostati della
forma separabile $\Psi(\bm{R}, \bm{r}) = \Phi(\bm{R})\psi(\bm{r})$, con
$\psi(\bm{r})$ tale da soddisfare la seguente equazione di Schrödinger
stazionaria
\begin{equation}
  \label{eq:schrodinger-stazionaria}
  \bigg(-\frac{\hslash^{2}}{2m}\nabla^{2} + V(\bm{r})\bigg)\psi(\bm{r})
  = E\psi(\bm{r}).
\end{equation}
Il moto del centro di massa, descritto dalla funzione d'onda $\Phi(\bm{R})$, non
è di nostro interesse poiché la sua hamiltoniana,
$-\hslash^{2}\nabla_{\bm{R}}^{2}/2(m_{1} + m_{2})$, è semplicemente quella di
particella libera, dunque ci occuperemo di studiare solo il moto relativo fra la
particella incidente e il bersaglio.  In pratica equivale a supporre che il
proiettile abbia massa $m$ e che il bersaglio sia infinitamente pesante e a
riposo nell'origine del sistema di riferimento.

\subsection{Forma asintotica degli stati stazionari}
\label{sec:forma-asintintotica}
% TODO (importante!): decidere una volta per tutte se la funzione d'onda dipende
% dal vettore d'onda o solo dal numero d'onda, similmente se l'ampiezza di
% diffusione dipende solo dal numero d'onda o da entrambi i momenti (iniziale e
% finale).  Spiegare le conclusioni nel posto opportuno e individuare una
% notazione coerente e possibilmente non troppo pesante.

Siamo interessati a individuare una forma asintotica per la funzione d'onda di
diffusione, cioè lontano dall'influenza del potenziale di interazione, poiché
negli esperimenti i rivelatori vengono posti molto lontani dal centro diffusore.
Molto tempo prima di raggiungere il bersaglio, il proiettile si muove come una
particella libera perché per valori sufficientemente grandi della distanza
relativa $r$ il potenziale $V(\bm{r})$ è praticamente nullo.  Allora la funzione
d'onda conterrà un termine di onda piana del tipo
$\e^{\uimm \bm{k}\cdot\bm{r}}$, con $\bm{k}$ vettore d'onda associato alla
particella di massa $m$, il cui modulo è legato all'energia $E$ da
$E = \hslash^{2}k^{2}/2m$.  Nella vicinanza del bersaglio, la funzione d'onda
della particella subirà una profonda modifica a causa dell'interazione con il
diffusore.  Tuttavia molto tempo dopo l'interazione, l'onda sarà lontano
dall'influenza del potenziale e la sua funzione d'onda avrà raggiunto una forma
più semplice: essa sarà la sovrapposizione di una funzione d'onda trasmessa
$\psi_{k,a}$ che continua a propagarsi nella direzione $\versor{k}$, quindi ha
la forma $\e^{\uimm \bm{k}\cdot\bm{r}}$, e una funzione d'onda di diffusione
$\psi_{k,b}(\bm{r})$
\begin{equation}
  \psi_{k}(\bm{r}) = \psi_{k,a}(\bm{r}) + \psi_{k,b}(\bm{r})
\end{equation}
Nei pedici delle funzioni d'onda abbiamo messo l'esplicita dipendenza
dall'energia attraverso il numero d'onda $k$.

L'espressione esplicita dell'onda di diffusione $\psi_{k,b}(\bm{r})$ dipende dal
particolare potenziale di interazione considerato, ma facendo un'analogia con
l'ottica ondulatoria possiamo prevedere una struttura generale per il suo
comportamento asintotico, vale a dire per grandi valori di $r$:
\begin{itemize}
\item in una fissata direzione angolare $(\theta, \phi)$ la funzione
  $\psi_{k,b}(\bm{r})$ avrà la forma di un'onda sferica uscente del tipo
  $\e^{\uimm kr}/r$ con la stessa energia dell'onda incidente
  $\e^{\uimm \bm{k}\cdot\bm{r}}$.  Il fattore $1/r$ assicura che il flusso della
  densità di probabilità $\abs{\psi_{k,b}}^{2}$ sia costante per ogni superficie
  sferica centrata nel bersaglio;
\item il processo di diffusione non è in generale isotropico, quindi l'ampiezza
  dell'onda di diffusione sarà modulata da un fattore $f_{k}(\theta,\phi)$
  dipendente dall'energia mediante il numero d'onda $k$ e dalle due coordinate
  angolari sferiche $\theta$ e
  $\phi$.\footnote{L'ampiezza di diffusione $f$ dipende, oltre che dalle
    variabili angolari $\theta$ e $\phi$, dal numero d'onda solo perché stiamo
    considerando diffusioni elastiche, nelle quali il modulo del momento del
    proiettile è conservato.  Nel caso più generale $f$ dipenderà dal momento
    iniziale e finale, vedi per esempio
    l'equazione~\eqref{eq:ampiezza-generale}.}
\end{itemize}
In definitiva il comportamento asintotico della funzione d'onda
$\psi_{k}(\bm{r})$ sarà del tipo
\begin{equation}
  \label{eq:forma-asintotica}
  \psi_{k}(\bm{r}) = \psi_{k,a}(\bm{r}) + \psi_{k,b}(\bm{r}) \underset{r \to
    \infty}{\sim} A \bigg( \e^{\uimm \bm{k}\cdot\bm{r}} +
  f_{k}(\theta,\phi)\frac{\e^{\uimm kr}}{r} \bigg),
\end{equation}
con $A$ fattore di normalizzazione.  Nei paragrafi successivi vedremo che questa
forma asintotica della funzione d'onda in un problema di diffusione, ricavata
qui su basi intuitive, è ben giustificata sotto opportune ipotesi.
% TODO: figura 8.1, pagina 348, del Gottfried oppure la prima figura del
% Sakurai, più difficile da realizzare ma più significativa.

Un'onda piana non rappresenta una particella, o un fascio di particelle,
incidente fisicamente accettabile dal momento che ha estensione infinita nello
spazio e nel tempo.  Una descrizione più realistica della particella incidente
si ottiene considerando un pacchetto di onde con dimensione limitata del tipo
\begin{equation}
  \psi(\bm{r},t) = \int_{0}^{\infty} \e^{-\uimm \hslash k^{2} t/2m} g(k)
  \psi_{k}(\bm{r}) \dd k,
\end{equation}
in cui la funzione $g(k)$, che per semplicità può essere presa reale, ha un
picco intorno a un punto $k = k_{0}$ ed è praticamente nulla altrove.  Si può
verificare che il pacchetto $\psi(\bm{r},t)$ così costruito è soluzione
dell'equazione di Schrödinger e quindi descrive l'evoluzione temporale della
particella relativa, vedi~\textcite[910-911]{cohen:quantum-mechanics}.
L'approssimazione del pacchetto di onde con un'onda piana è accettabile se la
dimensione del pacchetto è molto più grande di quella del diffusore, oppure del
raggio d'azione del potenziale di diffusione.

\subsection{Sezione d'urto}
\label{sec:sez-urto-mq}

Lo stato quantico $\psi(\bm{r})$ di una particella non descrive esattamente la
sua posizione ma l'ampiezza della probabilità di trovare la particella nella
posizione $\bm{r}$.  Analogamente, il flusso che dobbiamo considerare per
calcolare la sezione d'urto differenziale è il flusso di probabilità, cioè la
probabilità per unità di tempo che la particella attraversi l'area unitaria.  È
noto che per una particella di massa $m$ nello stato $\psi$ il flusso di
probabilità vale
\begin{equation}
  \bm{J} = \frac{\hslash}{m} \Im(\psi^{*}\nabla\psi).
\end{equation}
Applicando questa equazione alla funzione d'onda
asintotica~\eqref{eq:forma-asintotica} abbiamo
\begin{equation}
  \label{eq:flusso}
  \bm{J} = \frac{\hslash}{m} \Im(\psi_{k,a}^{*}\nabla\psi_{k,a} +
  \psi_{k,a}^{*}\nabla\psi_{k,b} + \psi_{k,b}^{*}\nabla\psi_{k,a} +
  \psi_{k,b}\nabla\psi_{k,b}).
\end{equation}
Trascuriamo momentaneamente i termini di interferenza
$\psi_{k,a}^{*}\nabla\psi_{k,b}$ e $\psi_{k,b}^{*}\nabla\psi_{k,a}$ e
identifichiamo il flusso associato alla sola funzione d'onda incidente
$\psi_{k,a}$ con il flusso $J_{a}$ delle particelle incidenti e il flusso della
sola funzione d'onda diffusa $\psi_{k,b}$ con il flusso di particelle diffuse
$J_{b}$.  In particolare, $J_{a}$ è
\begin{equation}
  \bm{J}_{a} = \frac{\hslash}{m}\Im(\psi_{k,a}^{*}\nabla\psi_{k,a}) =
  \frac{\abs{A}^{2}\hslash \bm{k}}{m},
\end{equation}
mentre le componenti del flusso di diffusione $\bm{J}_{b}$ sono
\begin{subequations}
  \begin{align}
    (\bm{J}_{b})_{r} &=
    \frac{\hslash}{m}\Im\bigg(\psi_{k,b}^{*}\parder{\psi_{k,b}}{r}\bigg) =
    \abs{Af_{k}(\theta,\phi)}^{2} \frac{\hslash k}{mr^{2}}, \\
    (\bm{J}_{b})_{\theta} &= \frac{\hslash}{m} \frac{1}{r^{3}} \Re
    \bigg(\frac{1}{\uimm}
    f_{k}^{*}(\theta,\phi) \parder{}{\theta}f_{k}(\theta,\phi)\bigg), \\
    (\bm{J}_{b})_{\phi} &= \frac{\hslash}{m} \frac{1}{r^{3} \sin\theta}
    \Re\bigg(\frac{1}{i}
    f_{k}^{*}(\theta,\phi) \parder{}{\phi}f_{k}(\theta,\phi)\bigg).
  \end{align}
\end{subequations}
Poiché stiamo considerando il comportamento asintotico, grandi $r$, le
componenti angolari del flusso $(\bm{J}_{b})_{\theta}$ e $(\bm{J}_{b})_{\phi}$
sono trascurabili rispetto alla componente radiale $(\bm{J}_{b})_{\textup{r}}$ e
approssimiamo $J_{b} \approx (\bm{J}_{b})_{r}$.  Inserendo questi risultati
nell'equazione~\eqref{eq:sez-urto} troviamo che la sezione d'urto differenziale
è
\begin{equation}
  \label{eq:sez-urto-mq}
  \sigma(\theta,\phi) = \frac{J_{b}r^{2}}{J_{a}} = \abs{f_{k}(\theta,\phi)}^{2}.
\end{equation}
Dunque, nei casi in cui la forma asintotica~\eqref{eq:forma-asintotica} della
funzione d'onda è valida, la sezione d'urto differenziale è calcolabile con la
formula precedente.  La sezione d'urto differenziale è la quantità di maggior
interesse nei problemi di diffusione perché è quella misurabile sperimentalmente
e abbiamo trovato che è uguale al modulo quadro dell'ampiezza di diffusione
$f_{k}(\theta,\phi)$.  La sezione d'urto non dipende dal fattore di
normalizzazione $A$ e spesso nel seguito lo trascureremo ponendolo uguale a $1$.
Nei prossimi paragrafi studieremo due metodi differenti per calcolare l'ampiezza
di diffusione: il metodo delle onde parziali e l'approssimazione di Born.

Ritorniamo all'equazione~\eqref{eq:flusso} e riprendiamo la descrizione del
processo di diffusione in termini del pacchetto d'onda.  Il fascio incidente
prima dell'urto, che nella pratica ha larghezza
finita,\footnote{Negli esperimenti questa condizione può essere raggiunta, per
  esempio, convogliando il fascio in un diaframma di larghezza sufficientemente
  grande da evitare anche gli effetti diffrattivi.}
è diretto verso il bersaglio.  Dopo aver interagito con questo, sono presenti
due pacchetti: un pacchetto di onde piane, come se il bersaglio non ci fosse, e
un pacchetto di onde diffuse dal bersaglio in tutte le direzioni.  Il pacchetto
trasmesso è dato dall'interferenza fra questi due pacchetti.  A causa della sua
larghezza limitata, a distanze infinite dal bersaglio il pacchetto incidente ha
densità di probabilità non nulla solo nella direzione in avanti, cioè
$\theta = 0$ o comunque $\theta$ molto piccolo, nelle altre direzioni
l'interferenza fra i due pacchetti è assente.  Negli esperimenti, il rivelatore
che misura il flusso di particelle diffuse è generalmente posto in direzioni
diverse da quella in avanti, dunque non riceve particelle trasmesse.  In questo
modo si osservano solo pacchetti di onde diffuse, non è necessario prendere in
considerazione i termini di interferenza fra i due pacchetti e i risultati
sperimentali possono essere confrontati correttamente con la sezione
d'urto~\eqref{eq:sez-urto-mq}.  Se il rivelatore fosse posto nella direzione
$\theta = 0$, da un punto di vista pratico risulterebbe difficile misurare
separatamente i due flussi di particelle, mentre nel calcolo della sezione
d'urto bisognerebbe tener conto dei termini di interferenza fra i pacchetti di
onde incidente e diffuse in avanti.  Questa interferenza distruttiva assicura la
conservazione del flusso di probabilità o, equivalentemente, del numero di
particelle causando una diminuzione del flusso per $\theta = 0$ rispetto al
flusso del solo pacchetto incidente: le particelle che non sono diffuse in
avanti sono rimosse dal fascio incidente che, dopo aver superato il bersaglio,
avrà quindi un'ampiezza minore.
% Vedi anche Siegfrid Flügge, "Practical Quantum Mechanics", problema 80,
% soluzione alle pagine 208-10.  Calcola esplicitamente la sezione d'urto
% tenendo conto anche dei termini di interferenza e fa vedere che la sezione
% d'urto nella forma sigma = |f|^2 è comunque sensata.

% TODO: parlare da qualche parte delle relazioni fra queste grandezze, associate
% alla particella relativa, e le corrispondenti grandezze nel sistema di
% riferimento del laboratorio o del centro di massa.  Vedi che dice il
% Ballentine all'inizio di pagina 427, *forse* si riferisce a questo.

\section{Metodo delle onde parziali}
\label{sec:onde-parziali}

\subsection{Sviluppo in onde parziali}
\label{sec:sviluppo-onde}

Il primo metodo che studieremo è particolarmente utile nei casi in cui il
potenziale di interazione ha simmetria sferica, cioè dipende solo dal modulo $r$
della distanza relativa fra i due corpi: $V(\bm{r}) = V(r)$.  L'equazione di
Schrödinger stazionaria assume la forma
\begin{equation}
  (\nabla^{2} + k^{2} - U(r)) \psi_{k}(\bm{r}) = 0,
\end{equation}
con $U(r) = (2m/\hslash^{2})V(r)$.  Poiché l'hamiltoniana commuta con gli
operatori di momento angolare $\bm{L}^{2}$ e $L_{z}$ sappiamo che una soluzione
dell'equazione precedente è del tipo separabile
$R_{kl}(r)Y_{l}^{m}(\theta,\phi)$, in cui $Y_{l}^{m}(\theta,\phi))$ è
un'armonica sferica e $R_{kl}(r)$ è una funzione puramente radiale che può
essere posta nella forma $R_{kl}(r) = u_{kl}(r)/r$ con la condizione
$u_{kl}(0) = 0$.  La soluzione generale dell'equazione precedente, autostato
dell'hamiltoniana con autovalore di energia $E = \hslash^{2}k^{2}/2m$, sarà una
combinazione lineare delle soluzioni appena illustrate con somma su tutti i
possibili valori di momento angolare $l$ e terza componente $m$, ma con fissato
valore del numero d'onda $k$ poiché stiamo considerando gli stati stazionari
dell'hamiltoniana
\begin{equation}
  \label{eq:onde-parziali1}
  \psi_{k}(\bm{r}) = \sum_{l = 0}^{+\infty} \sum_{m = -l}^{l} \psi_{klm}(\bm{r})
  = \sum_{l = 0}^{+\infty} \sum_{m = -l}^{l} a_{lm}
  \frac{u_{kl}(r)}{r}Y_{l}^{m}(\theta,\phi).
\end{equation}
Ciascuna delle funzioni $\psi_{klm}$ prende il nome di \emph{onda parziale} e la
loro combinazione lineare è detta \emph{sviluppo in onde parziali}.  L'equazione
radiale che soddisfa $R_{kl}$ è
\begin{equation}
  \label{eq:diff-R-op}
  \bigg(\frac{1}{r^{2}}\toder{}{r}r^{2}\toder{}{r} + k^{2} - U(r) -
  \frac{l(l+1)}{r^{2}}\bigg)R_{kl}(r) = 0.
\end{equation}
Sostituendo $R_{kl}=u_{kl}/r$ troviamo che la funzione $u_{kl}$ soddisfa
un'equazione differenziale più semplice
\begin{equation}
  \label{eq:diff-u-op}
  \bigg(\toder[2]{}{r} + k^{2} - U(r) - \frac{l(l+1)}{r^{2}}\bigg)u_{kl}(r) = 0.
\end{equation}
Il potenziale di interazione è a simmetria sferica, la particella incidente
rompe la completa simmetria definendo una direzione precisa che identifichiamo
con l'asse $\versor{z}$ (cioè poniamo $\versor{z}$ tale che
$\bm{k} = k\versor{z}$), tuttavia non è presente alcuna dipendenza dall'angolo
azimutale $\phi$ e ci sarà pertanto simmetria cilindrica.  Per annullare la
dipendenza della funzione d'onda $\psi_{k}(\bm{r})$ da $\phi$ nello sviluppo in
onde parziali~\eqref{eq:onde-parziali1} dobbiamo considerare solo i termini con
$m = 0$ perché le armoniche sferiche dipende da $\phi$ attraverso
$\e^{\uimm m \phi}$.

% TODO: dimostrare lo sviluppo in onde parziali dell'onda piana
Tenendo anche presente quanto appena notato, possiamo ipotizzare una forma più
precisa per lo sviluppo in onde parziali di $\psi_{k}(\bm{r})$.  Partiamo
dall'osservare che un'onda piana, quindi in assenza di potenziale, con vettore
d'onda $\bm{k} = k\versor{z}$ può essere sviluppata in onde parziali nel
seguente modo (vedi~\textcite[928-929]{cohen:quantum-mechanics})
\begin{equation}
  \label{eq:sviluppo-onda-piana}
  \e^{\uimm \bm{k}\cdot\bm{r}} = \sum_{l} (2l+1) \uimm^{l} j_{l}(kr)
  P_{l}(\cos\theta),
\end{equation}
in cui $j_{l}$ è la funzione di Bessel sferica di ordine $l$, $P_{l}$ è il
polinomio di Legendre di grado $l$ e $\theta$ è l'angolo compreso fra
$\bm{k} = k\versor{z}$ e $\bm{r}$.

Supponiamo che il potenziale di interazione sia a \emph{rapida decrescenza}, o
\emph{a corto raggio}, vale a dire per $r$ tendente all'infinito va a $0$ più
rapidamente di $1/r^{2}$
\begin{equation}
  \lim_{r \to \infty} r^{2}V(r) = 0.
\end{equation}
In questo modo stiamo escludendo dalla trattazione il potenziale di Coulomb,
nonostante sia a simmetria sferica, perché decresce come $1/r$.  Con questa
ipotesi possiamo supporre che l'andamento asintotico della funzione d'onda di
diffusione stazionaria, con fissato valore dell'energia e di conseguenza di $k$,
sia del tipo
\begin{equation}
  \label{eq:onde-parziali2}
  \psi_{k}(\bm{r}) = \sum_{l} (2l+1)\uimm^{l} A_{l} R_{kl}(r) P_{l}(\cos\theta),
\end{equation}
Rispetto al caso di potenziale nullo stiamo dunque assumendo che al posto delle
funzioni di Bessel sferiche ci siano le funzioni $R_{kl}$ precedentemente
introdotte e inoltre dei coefficienti $A_{l}$ da determinare.

\subsection{Sfasamenti}
\label{sec:sfasamenti}

\begin{figure}
  \centering
  \begin{tikzpicture}[scale=0.7,font=\footnotesize]
    \draw[fill=white!90!black](0,0) circle (4);
    \node[circle,draw,align=center,fill=white!80!black] (0,0)
    {Regione di\\diffusione\\ $V \neq 0$};
    \node[align=center] at (0,2.6) {Regione intermedia\\ $V\approx0$};
    \node[align=center] at (10,1.5) {Zona di radiazione\\ $kr \gg 1$};
  \end{tikzpicture}
  \caption{Diffusione da potenziale a corto raggio con simmetria sferica.  Nelle
    immediate vicinanze del bersaglio, $r \approx 0$, si ha la
    \emph{regione di diffusione} nella quale il potenziale non è trascurabile.
    All'aumentare della distanza dal bersaglio il potenziale centrifugo
    $\hslash^2 l(l+1)/2mr^2$ domina su $V(r)$ e questa condizione determina la
    \emph{regione intermedia}.  La \emph{zona di radiazione} si trova a grande
    distanza dal bersaglio, quindi $kr \gg 1$ e sia il potenziale di interazione
    sia il potenziale centrifugo sono trascurabili rispetto a $k^2$.}
\label{fig:regioni-potenziale-sferico}
\end{figure}
Sotto l'ipotesi di potenziale a corto raggio, con riferimento alla
figura~\ref{fig:regioni-potenziale-sferico} possiamo suddividere lo spazio in
tre regioni: la regione di diffusione, in cui il potenziale di interazione è
sensibilmente diverso da zero, la regione intermedia in cui $U \ll l(l+1)/r^{2}$
e la zona di radiazione, in cui $kr \gg 1$ e anche il termine di momento
angolare è trascurabile nell'hamiltoniana.  Nella zona di radiazione, cioè per
grandi valori di $r$, l'equazione di Schrödinger radiale~\eqref{eq:diff-u-op} si
riduce a
\begin{equation}
  \toder[2]{u_{kl}}{r^{2}} = -k^{2}u.
\end{equation}
La soluzione generale è data da
\begin{equation}
  u_{kl}(r) = D\e^{\uimm kr} + F\e^{-\uimm kr}.
\end{equation}
Il primo termine rappresenta un'onda sferica uscente, il secondo un'onda sferica
entrante.  Nel problema di diffusione è presente solo l'onda sferica uscente,
quindi $F = 0$ e
\begin{equation}
  R_{kl}(r) \sim \frac{\e^{\uimm kr}}{r}
\end{equation}
come avevamo previsto nella forma asintotica~\eqref{eq:forma-asintotica}.

\begin{figure}
  \centering
  \subfloat[][\emph{Funzioni di Bessel sferiche di prima specie}, o
  semplicemente \emph{funzioni di Bessel sferiche}, degli ordini più
  bassi.]{\input{gnuplot/bessel}}
  \\
  \subfloat[][\emph{Funzioni di Bessel sferiche di seconda specie}, chiamate
  anche \emph{funzioni di Neumann sferiche}, degli ordini più
  bassi.]{\input{gnuplot/neumann}}
  \caption{Andamenti delle prime funzioni di Bessel sferiche.}
  \label{fig:bessel}
\end{figure}
Nella regione intermedia l'equazione radiale~\eqref{eq:diff-u-op} è
\begin{equation}
  \toder[2]{u_{kl}}{r} - \frac{l(l+1)}{r^{2}}u_{kl} = -k^{2}u_{kl}.
\end{equation}
La soluzione di questa equazione è data dalla combinazione lineare delle
funzioni di Bessel sferiche $j_{l}$ e delle funzioni di Neumann sferiche $n_{l}$
\begin{equation}
  u_{kl}(r) = Brj_{l}(kr) + Crn_{l}(kr) \implies R_{kl}(r) = Bj_{l}(kr) +
  Cn_{l}(kr).
\end{equation}
Le funzioni sferiche di Bessel e di Neumann hanno i seguenti comportamenti
asintotici
\begin{subequations}
  \label{eq:asintoti-bessel}
  \begin{align}
    \label{eq:jl-asintotico-orig}
    j_{l}(\rho) &\underset{\rho \to 0}{\sim} \frac{\rho^{l}}{(2l+1)!!}, \\
    \label{eq:jl-asintotico-inf}
    j_{l}(\rho) &\underset{\rho \to \infty}{\sim} \frac{1}{\rho} \sin\bigg(\rho
    - l \frac{\pi}{2}\bigg), \\
    n_{l}(\rho) &\underset{\rho \to 0}{\sim} \frac{(2l-1)!!}{\rho^{l+1}}, \\
    n_{l}(\rho) &\underset{\rho \to \infty}{\sim} -\frac{1}{\rho}\cos\bigg(\rho
    - l \frac{\pi}{2}\bigg).
  \end{align}
\end{subequations}
Gli andamenti delle funzioni di Bessel e di Neumann sferiche degli ordini più
bassi sono rappresentati nella figura~\ref{fig:bessel}.  Nell'origine le
funzioni di Bessel convergono, mentre le funzioni di Neumann divergono.  Per
normalizzare $R_{kl}$ scegliamo i coefficienti $B$ e $C$ tali che
$\abs{B}^{2} + \abs{C}^{2} = 1$, in particolare poniamo $B = \cos\delta_{l}$ e
$C = -\sin\delta_{l}$, da cui
\begin{equation}
  \label{eq:R-intermedio}
  R_{kl}(r) = j_{l}(kr)\cos\delta_{l} - n_{l}(kr)\sin\delta_{l}.
\end{equation}
L'equazione differenziale che soddisfa $R_{kl}$ è reale, la soluzione può essere
scelta reale e anche i $\delta_{l}$ dovranno essere reali.  Con questa
posizione, usando le proprietà~\eqref{eq:asintoti-bessel} troviamo che il
comportamento asintotico per grandi valori di $kr$, cioè nella zona di
radiazione, di $R_{kl}$ è
\begin{equation}
  \label{eq:R-asintotico}
  R_{kl}(r) \underset{kr \to \infty}{\sim} \frac{\sin(kr - l\pi/2 +
    \delta_{l})}{kr}.
\end{equation}
Se non ci fosse potenziale di interazione, l'espressione~\eqref{eq:R-intermedio}
della funzione radiale sarebbe valida fino a $r=0$, non solo nella regione
intermedia.  Abbiamo osservato che le funzioni di Neumann sferiche nell'origine
divergono come $1/r^{l+1}$, ma la parte radiale della funzione d'onda non può
avere questo comportamento, dunque deve essere $\delta_{l} = 0$ per ogni $l$ e
$r$ in assenza di potenziale e in questo caso si avrebbe $R_{kl}(r) =
j_{l}(kr)$.
Questo risultato supporta lo sviluppo in onde parziali ipotizzato
nell'equazione~\eqref{eq:onde-parziali2}.  A questo punto possiamo confrontare
l'andamento asintotico~\eqref{eq:R-asintotico} di $R_{kl}(r)$ con
quello~\eqref{eq:jl-asintotico-inf} del caso di potenziale nullo e riconosciamo
che l'effetto di un potenziale sferico a corto raggio a grandi distanze $r$ è
quello di introdurre uno sfasamento in ciascuna funzione radiale asintotica
$R_{kl}(r)$ dello sviluppo in onde parziali~\eqref{eq:onde-parziali2}.

\subsection{Calcolo della sezione d'urto}
\label{sec:sez-urto-op}

Nel paragrafo~\ref{sec:sez-urto-mq} abbiamo visto che la sezione d'urto
differenziale è determinata dal comportamento asintotico della funzione d'onda.
I potenziali sferici a corto raggio introducono uno sfasamento negli stati
stazionari di diffusione, ci aspettiamo che in questo caso la sezione d'urto
differenziale sia esprimibile in funzione degli sfasamenti.

Sostituiamo gli sviluppi in onde parziali~\eqref{eq:sviluppo-onda-piana} e
\eqref{eq:onde-parziali2} nell'espressione
asintotica~\eqref{eq:forma-asintotica} della funzione d'onda, con $A = 1$ per
semplicità, ricordando i comportamenti asintotici~\eqref{eq:jl-asintotico-inf} e
\eqref{eq:R-asintotico}
\begin{multline}
  \label{eq:foo}
  \sum_{l}(2l+1)\uimm^{l}P_{l}(\cos\theta)A_{l}\frac{\sin(kr - l\pi/2 +
    \delta_{l})}{kr} \\
  = \sum_{l}(2l+1)\uimm^{l} P_{l}(\cos\theta)\frac{\sin(kr - l\pi/2)}{kr} +
  f_{k}(\theta,\phi)\frac{\e^{\uimm kr}}{r}.
\end{multline}
Usando la relazione $\sin x = (\e^{\uimm x} - \e^{-\uimm x})/2\uimm$ e
uguagliando i coefficienti di $\e^{-\uimm kr}$ in ambo i membri così ottenuti
troviamo
\begin{multline}
  \sum_{l}(2l+1)\uimm^{l}P_{l}(\cos\theta)A_{l}\exp(\uimm l\pi/2 -
  \uimm\delta_{l}) \\
  = \sum_{l}(2l+1)\uimm^{l}P_{l}(\cos\theta)\exp(\uimm l\pi/2).
\end{multline}
I polinomi di Legendre sono funzioni linearmente indipendenti e affinché
l'equazione precedente sia valida devono essere uguali i coefficienti dei
$P_{l}$ dello stesso grado $l$, da cui ricaviamo che
\begin{equation}
  A_{l} = \e^{\uimm \delta_{l}}.
\end{equation}
Procedendo in maniera analoga, uguagliando i coefficienti di $\e^{\uimm kr}$
nella~\eqref{eq:foo} e ricordando il risultato appena determinato troviamo
l'ampiezza di diffusione
\begin{equation}
  \label{eq:ampiezza-diffusione-op}
  \begin{split}
    f_{k}(\theta,\phi) &= f_{k}(\theta) = \frac{1}{2\uimm k}\sum_{l}(2l+1)
    \overbrace{\uimm^{l}\e^{-\uimm l\pi/2}}^{\uimm^{l}(-\uimm)^{l}=1}
    (\e^{2\uimm\delta_{l}} - 1)P_{l}(\cos\theta) \\
    &= \frac{1}{2\uimm k}\sum_{l}(2l+1)(\e^{2\uimm\delta_{l}} - 1)
    P_{l}(\cos\theta) \\
    &= \frac{1}{k} \sum_{l} (2l+1) \sin\delta_{l} \e^{\uimm\delta_{l}}
    P_{l}(\cos\theta).
  \end{split}
\end{equation}
L'ampiezza di diffusione dipende solo dalla colatitudine $\theta$ perché, come
notato in precedenza, nelle diffusioni da campi sferici c'è simmetria
cilindrica.  Inoltre $f_{k}(\theta)$ non cambia per effetto di una sostituzione
$\delta_{l} \to \delta_{l} + \pi$.

Poiché abbiamo verificato che la forma asintotica~\eqref{eq:forma-asintotica}
della funzione d'onda è valida, possiamo calcolare la sezione d'urto
differenziale con l'equazione~\eqref{eq:sez-urto-mq}
\begin{equation}
  \sigma(\theta,\phi) = \sigma(\theta) = \abs{f_{k}(\theta)}^{2}.
\end{equation}
La sezione d'urto totale si ricava integrando la sezione d'urto differenziale su
tutto l'angolo solido.  I polinomi di Legendre sono ortogonali, infatti
soddisfano la seguente relazione
\begin{equation}
  \int_{-1}^{1}P_{l}(u)P_{l'}(u)\dd u = \frac{2\delta_{ll'}}{2l+1}
\end{equation}
e da questa ricaviamo
\begin{equation}
  \label{eq:serie-sfasamenti}
  \begin{split}
    \sigma_{\textup{tot}} &= \int_{0}^{2\pi}\dd \phi \int_{0}^{\pi}
    \abs{f_{k}(\theta)}^{2} \sin\theta \dd\theta \\
    &= 2\pi \frac{1}{k^{2}} \sum_{ll'}(2l+1)(2l'+1) \sin\delta_{l} \e^{-\uimm
      \delta_{l}} \sin\delta_{l'}\e^{\uimm \delta_{l'}}
    \int_{-1}^{1}P_{l}(\cos\theta)P_{l'}(\cos\theta) \dd(\cos\theta) \\
    &= \frac{4\pi}{k^{2}} \sum_{l}(2l+1)\sin^{2}\delta_{l}.
  \end{split}
\end{equation}
Utilizzando le altre espressioni dell'ampiezza di diffusione $f_{k}(\theta)$ si
possono trovare con calcoli analoghi espressioni differenti per la sezione
d'urto totale
\begin{equation}
  \sigma_{\textup{tot}} = \sum_{l} \sigma_{l} = \frac{\pi}{k^{2}}
  \sum_{l}(2l+1)\abs{1 - S_{l}}^{2} = \frac{2\pi}{k^{2}}
  \sum_{l}(2l+1)(1-\Re(S_{l})),
\end{equation}
con $S_{l} = \e^{2\uimm\delta_{l}}$.  Infine notiamo che ponendo $\theta = 0$
nell'ultimo membro dell'equazione~\eqref{eq:ampiezza-diffusione-op} e ricordando
che $P_{l}(1) = 1$ abbiamo
\begin{equation}
  f_{k}(0) = \frac{1}{k} \sum_{l}(2l+1)(\sin\delta_{l}\cos\delta_{l} + \uimm
  \sin^{2}\delta_{l}),
\end{equation}
quindi
\begin{equation}
  \sigma_{\textup{tot}} = \frac{4\pi}{k} \Im(f_{k}(0)).
\end{equation}
Quest'ultimo risultato è conosciuto come \emph{teorema ottico} e ci dice che la
sezione d'urto totale è determinata dall'ampiezza di diffusione elastica in
avanti, cioè per $\theta=0$.  In questa direzione c'è la sovrapposizione e
interferenza tra il pacchetto di onde incidenti e onde diffuse elasticamente e
questa causa la rimozione di flusso dal fascio di particelle incidenti, come
discusso alla fine del paragrafo~\ref{sec:sez-urto-mq}.  La sezione d'urto
totale quantifica proprio questa rimozione e il teorema ottico rendo conto della
conservazione del flusso di probabilità totale.  L'aggettivo ``ottico'' è dovuto
al fatto che in ottica l'interferenza fra un'onda incidente e l'onda diffusa in
avanti è l'origine dell'ombra di un oggetto opaco.  Anche se abbiamo dimostrato
il teorema ottico solo nel caso di diffusione elastica da potenziale sferico,
esso vale anche per processi di diffusione anelastica da altri tipi di
potenziali.  In questo caso la sezione d'urto totale è la somma di quelle dei
canali elastico e anelastico, l'ampiezza di diffusione in avanti che compare al
secondo membro è solo quella elastica.

Se il potenziale $U(r)$ fosse identicamente nullo, avremmo
$R_{kl}(r) = j_{l}(kr)$ e dall'equazione~\eqref{eq:jl-asintotico-orig} vediamo
che per $kr \ll l$ va come $(kr)^{l}$ e diventa sempre più piccolo al crescere
di $l$.  Se $U(r)$ è a corto raggio, in particolare con raggio uguale a $L$,
nell'equazione~\eqref{eq:diff-R-op} il potenziale è moltiplicato per la piccola
quantità $R_{kl}(r)$ e avrà un piccolo effetto sulla soluzione.  Questo rozzo
ragionamento ci permette di intuire che gli sfasamenti $\delta_{l}$ sono piccoli
se $kL \ll l$ e in questo caso la serie~\eqref{eq:serie-sfasamenti} converge già
dopo pochi termini e risulta maggiormente utile a livello pratico.  In molti
casi già il termine con $l = 0$ è sufficiente a dare una buona stima della
sezione d'urto e in questi casi si parla di diffusione in onda $s$.  La
condizione $kL \ll l$ è vera, indipendentemente dal valore del raggio $L$ del
potenziale, per piccoli valori dell'energia della particella incidente.  D'altra
parte potevamo aspettarci questo risultato perché se il potenziale è centrale e
l'energia della particella incidente è bassa, la simmetria sferica è poco
perturbata e nello sviluppo in onde parziali domina il termine di onda $s$, la
quale è isotropica.

% TODO: Sakurai, pagg 400-402, e Gottfried, pagg 336-8, ricavano l'ampiezza di
% diffusione in maniera diversa, senza risolvere l'equazione di Schroëdinger.
% Valutare se metterla, magari in aggiunta a quella che ho fatto io e in
% {\small [...]}.

\subsection{Calcolo degli sfasamenti}
\label{sec:calcolo-sfasamenti}

Se il potenziale è non nullo per una distanza relativa $r < L$ ed è assente per
distanze maggiori è possibile trovare una formula che fornisce direttamente gli
sfasamenti $\delta_{l}$.  Bisogna risolvere, eventualmente con metodi numerici,
l'equazione differenziale~\eqref{eq:diff-R-op} nella regione di diffusione
$r < L$ ed effettuare il raccordo con la forma~\eqref{eq:R-intermedio} valutata
nel punto $r = L$.  Poiché la~\eqref{eq:diff-R-op} è un'equazione differenziale
del secondo ordine, ci saranno due soluzioni linearmente indipendenti, di cui
solo una non diverge per nell'origine, come succede per esempio in assenza di
potenziale con le funzioni sferiche di Bessel e Neumann.  Per determinare gli
sfasamenti possiamo imporre la continuità in $r=L$ di $R_{kl}$ e
$\ltoder{R_{kl}}{r}$, ma è più conveniente imporre che sia continua da sinistra
in $r=L$ la derivata logaritmica
\begin{equation}
  \gamma_{kl} = \toder{\log R_{kl}}{r} = \frac{1}{R_{kl}} \toder{R_{kl}}{r},
\end{equation}
che ha il vantaggio di essere indipendente dalla normalizzazione della funzione
radiale.  Sostituiamo l'espressione~\eqref{eq:R-intermedio} di $R_{kl}$, valida
nella regione intermedia, nell'equazione precedente e valutiamo il risultato nel
punto $r = L$
\begin{equation}
  \gamma_{kl} = \frac{k(j'_{l}(kL)\cos\delta_{l} -
    n'_{l}(kL)\sin\delta_{l})}{j_{l}(kL)\cos\delta_{l} -
    n_{l}(kL)\sin\delta_{l}},
\end{equation}
in cui $j'_{l}(kL)$ indica la derivata di $j_{l}(kr)$ rispetto a $kr$ e valutata
nel punto $kr = kL$.  Significato analogo per $n'_{l}(kL)$.  Riarrangiando
l'equazione precedente è possibile ricavare la tangente di $\delta_{l}$
\begin{equation}
  \label{eq:tangente-sfasamento}
  \tan\delta_{l} = \frac{kj'_{l}(kL) - \gamma_{kl}j_{l}(kL)}{kn'_{l}(kL) -
    \gamma_{kl}n_{l}(kL)}.
\end{equation}
Si definiscono le funzioni di Hankel sferiche di prima e seconda specie
\begin{subequations}
  \begin{align}
    h_{l}^{(1)}(\rho) &= j_{l}(\rho) + \uimm n_{l}(\rho), \\
    h_{l}^{(2)}(\rho) &= j_{l}(\rho) - \uimm n_{l}(\rho)
  \end{align}
\end{subequations}
e vediamo come possono essere utilizzate per calcolare gli sfasamenti.  Tramite
gli sviluppi in esponenziali delle funzioni seno e coseno si trova la seguente
relazione valida in generale
\begin{equation}
  \e^{2\uimm x} = \frac{\uimm \tan x + 1}{1 - \uimm \tan x},
\end{equation}
da cui
\begin{equation}
  S_{l} = \e^{2\uimm\delta_{l}} = -\frac{R_{kl}(L)k h_{l}^{(2)}{}'(kL) -
    R'_{kl}(L)h_{l}^{(2)}(kL)}{R_{kl}(L)k h_{l}^{(1)}{}'(kL) -
    R'_{kl}(L)h_{l}^{(1)}(kL)} = -\left.\frac{W[R_{kl}(r),
      h_{l}^{(2)}(kr)]}{W[R_{kl}(r), h_{l}^{(1)}(kr)]}\right|_{r = L}.
\end{equation}
Il simbolo $W[\cdot,\cdot]$ indica il wronskiano delle due funzioni, con
derivate rispetto alla coordinata $r$, dunque per le funzioni di Hankel si ha
$\partial_{r}h(kr) = k \partial_{\rho}h(\rho) = k h'(\rho)$.

Se il potenziale non è esattamente nullo per $r > L$ ma è comunque a corto
raggio, possiamo calcolare gli sfasamenti usando ancora la
formula~\eqref{eq:tangente-sfasamento} nel limite $L \to \infty$, ricordando che
anche $\gamma_{kl}$ dipende da $L$.  Infine si può dimostrare che
\begin{equation}
  \delta_{l} \underset{l \to \infty}{\sim} \frac{1}{l!}.
\end{equation}

\section{Metodo dell'approssimazione di Born}
\label{sec:metodo-born}

Abbiamo visto che il metodo dello sviluppo in onde parziali è utile nei processi
a bassa energia.  Il metodo dell'approssimazione di Born che andiamo a
presentare è più utile, invece, nel caso di alte energie delle particelle
incidenti.

\subsection{Equazione integrale di diffusione}
\label{sec:equazione-integrale}

L'equazione di Schrödinger stazionaria~\eqref{eq:schrodinger-stazionaria} può
essere riscritta nella forma
\begin{equation}
  \label{eq:helmholtz}
  (\nabla^{2} + k^{2})\psi_{k}(\bm{r}) = Q(\bm{r}),
\end{equation}
in cui
\begin{equation}
  Q(\bm{r}) = U(\bm{r})\psi_{k}(\bm{r}) =
  \frac{2m}{\hslash^{2}}V(\bm{r})\psi_{k}(\bm{r}).
\end{equation}
L'equazione~\eqref{eq:helmholtz} assomiglia all'equazione di Helmholtz non
omogenea, qui però il termine noto $Q(\bm{r})$ dipende a sua volta
dall'incognita $\psi_{k}(\bm{r})$.
% TODO: eventualmente fare tutti i passaggi per ricavare la funzione di Green
Dalla teoria delle funzioni di Green si sa che la soluzione generale
dell'equazione~\eqref{eq:helmholtz} è data da
\begin{equation}
  \begin{split}
    \psi_{k}(\bm{r}) &= \psi_{0,k}(\bm{r}) + \int
    G(\bm{r},\bm{r}_{0};k)Q(\bm{r}_{0})\dd^{3}\bm{r}_{0} \\
    &= \psi_{0,k}(\bm{r}) + \frac{2m}{\hslash^{2}}\int
    G(\bm{r},\bm{r}_{0};k)V(\bm{r}_{0})\psi_{k}(\bm{r}_{0})\dd^{3}\bm{r}_{0},
  \end{split}
\end{equation}
in cui $G(\bm{r},\bm{r}';k)$ è una funzione di Green dell'operatore
$\nabla^{2} + k^{2}$, cioè è la soluzione dell'equazione
\begin{equation}
  \label{eq:green}
  (\nabla^{2} + k^{2})G(\bm{r},\bm{r}';k) = \delta^{3}(\bm{r} - \bm{r}'),
\end{equation}
mentre $\psi_{0,k}(\bm{r})$ è una soluzione particolare dell'equazione di
Helmholtz omogenea
\begin{equation}
  (\nabla^{2} + k^{2})\psi_{0,k}(\bm{r}) = 0.
\end{equation}
Quest'ultima equazione non è altro che l'equazione di Schrödinger stazionaria di
particella libera, quindi $\psi_{0,k}(\bm{r}) = A\e^{\uimm \bm{k}\cdot\bm{r}}$,
con $A$ costante di normalizzazione.  Due soluzioni
dell'equazione~\eqref{eq:green} sono le funzioni di Green uscente
$G_{+}(\bm{r},\bm{r}';k)$ ed entrante $G_{-}(\bm{r},\bm{r}';k)$ definite da
(vedi~\textcite[450-452]{ballentine:quantum-mechanics},
\textcite[408-411]{griffiths:introduction-qm})
\begin{equation}
  G_{\pm}(\bm{r},\bm{r}';k) = -\frac{\e^{\pm \uimm k\abs{\bm{r} -
        \bm{r}'}}}{4\pi \abs{\bm{r} - \bm{r}'}}.
\end{equation}
Poiché vogliamo che il comportamento asintotico della funzione d'onda sia del
tipo~\eqref{eq:forma-asintotica}, si intuisce che dobbiamo prendere in
considerazione la funzione di Green uscente $G_{+}$ e in effetti verificheremo
più avanti che questa scelta soddisfa la nostra richiesta.  In definitiva la
soluzione dell'equazione~\eqref{eq:helmholtz} è data da
\begin{equation}
  \label{eq:lippmann-schwinger}
  \psi_{k}(\bm{r}) = A\e^{\uimm \bm{k}\cdot\bm{r}} - \frac{m}{2\pi\hslash^{2}}
  \int \frac{\e^{\uimm k\abs{\bm{r} - \bm{r}_{0}}}}{\abs{\bm{r} - \bm{r}_{0}}}
  V(\bm{r}_{0}) \psi_{k}(\bm{r}_{0}) \dd^{3}\bm{r}_{0}.
\end{equation}
Questa è l'\emph{equazione integrale di diffusione}, chiamata anche
\emph{equazione di Lippmann–Schwinger}.  Essa è equivalente all'equazione di
Schrödinger stazionaria~\eqref{eq:schrodinger-stazionaria} ma in più contiene al
suo interno la condizione al contorno data dal termine
$\psi_{0,k}(\bm{r}) = A\e^{\uimm \bm{k}\cdot\bm{r}}$ che descrive il
comportamento della funzione d'onda in assenza di potenziale.

\subsection{Serie di Born}
\label{sec:serie-born}

Poniamo il vettore d'onda del flusso diffuso uguale a $\bm{k}' = k\versor{r}$
perché per la conservazione dell'energia ha lo stesso modulo $k$ del vettore
d'onda incidente $\bm{k} = k\versor{z}$.  Supponiamo anche in questo caso che il
potenziale sia a corto raggio, cioè che risulti
$\lim_{r \to \infty} r^{2}V(\bm{r}) = 0$.  Poiché inoltre vogliamo determinare
$\psi_{k}(\bm{r})$ molto lontano dalla regione di diffusione, cioè
$\abs{\bm{r}} \gg \abs{\bm{r}_{0}}$, possiamo utilizzare le approssimazioni, con
$\alpha$ angolo compreso fra $\bm{r}$ e $\bm{r}_{0}$,
\begin{subequations}
  \begin{align}
    \begin{split}
      \abs{\bm{r} - \bm{r}_{0}} &=
      r\bigg(1-2\frac{r_{0}}{r}\cos\alpha+\frac{r_{0}^{2}}{r^{2}}\bigg)^{1/2}
      \approx r - \versor{r}\cdot\bm{r}_{0},
    \end{split} \\
    \begin{split}
      \frac{1}{\abs{\bm{r} - \bm{r}_{0}}} &= \frac{1}{r}\bigg(1 -
      2\frac{r_{0}}{r}\cos\alpha + \frac{r_{0}^{2}}{r^{2}}\bigg)^{-1/2} \approx
      \frac{1}{r} + \frac{\bm{r}\cdot\bm{r}_{0}}{r^{2}} \approx \frac{1}{r}.
    \end{split}
  \end{align}
\end{subequations}
La funzione di Green uscente può allora essere approssimata con
\begin{equation}
  G_{+}(\bm{r},\bm{r}';k) = \frac{\e^{\uimm k\abs{\bm{r} -
        \bm{r}_{0}}}}{\abs{\bm{r} - \bm{r}_{0}}} \approx \frac{\e^{\uimm kr}}{r}
  \e^{-\uimm \bm{k}'\cdot\bm{r}_{0}}.
\end{equation}
Abbiamo trovato che per grandi valori di $r$ la funzione d'onda è data da
\begin{equation}
  \label{eq:bar}
  \psi_{k}(\bm{r}) = A\e^{\uimm \bm{k}\cdot\bm{r}} - \frac{\e^{\uimm kr}}{r}
  \frac{m}{2\pi\hslash^{2}} \int \e^{-\uimm \bm{k}'\cdot\bm{r}_{0}}
  V(\bm{r}_{0}) \psi_{k}(\bm{r}_{0}) \dd^{3}\bm{r}_{0} = A\bigg(\e^{\uimm
    \bm{k}\cdot\bm{r}} + f_{k}(\theta,\phi)\frac{\e^{\uimm kr}}{r}\bigg),
\end{equation}
con
\begin{equation}
  f_{k}(\theta,\phi) = -\frac{m}{2\pi\hslash^{2}A} \int\e^{-\uimm
    \bm{k}'\cdot\bm{r}_{0}} V(\bm{r}_{0}) \psi_{k}(\bm{r}_{0})\dd^{3}\bm{r}_{0}.
\end{equation}
Questo risultato conferma anche in questo caso la forma
asintotica~\eqref{eq:forma-asintotica} della funzione d'onda di diffusione.
D'ora in poi porremo, per semplicità, il coefficiente di normalizzazione $A=1$.
Nella rappresentazione bra-ket l'ampiezza di diffusione può essere scritta come
\begin{equation}
  \label{eq:ampiezza-braket}
  f_{k}(\theta,\phi) = -\frac{m}{2\pi\hslash^{2}} \braket{\psi_{0,\bm{k}'} | V |
    \psi_{k}} = -\frac{m}{2\pi\hslash^{2}} \int\e^{-\uimm
    \bm{k}'\cdot\bm{r}_{0}} V(\bm{r}_{0}) \psi_{k}(\bm{r}_{0}) \dd^{3}\bm{r}_{0}
\end{equation}
in cui $\psi_{0,\bm{k'}} = \e^{\uimm \bm{k}'\cdot\bm{r}}$ è autostato
dell'hamiltoniana $H_{0} = -\hslash^{2}\nabla^{2}/2m$.

\begin{figure}
  \centering
  {\footnotesize
    $ \displaystyle
    \psi =
    \tikz[baseline=\allinea]{
      \draw[middlearrow={latex}] (0,0) -- node[below] {$\psi_{0}$} (1.8,0);} +
    \tikz[baseline=\allinea]{
      \draw[middlearrow={latex}] (0,0) -- node[below] {$\psi_{0}$} (2.4,0)
      node[shape=coordinate] (V) {};
      \draw (V) circle (1);
      \draw (V) node[below] {$V$} -- node[above right,sloped] {$g$} ++(1.5,1.2);
      \node[dot] at (V) {};} +
    \tikz[baseline=\allinea]{
      \draw[middlearrow={latex}] (0,0) -- node[below]
      {$\psi_{0}$} (1.4,0) node[shape=coordinate] (V1) {};
      \draw (V1) node[below] {$V$} -- node[above,sloped] {$g$} ++(0.9,0.3)
      node[shape=coordinate] (V2) {};
      \draw (V2) node[below] {$V$} -- node[above right,sloped] {$g$}
      ++(1.3,-0.4);
      \draw (1.9,0) circle (1);
      \node[dot] at (V1) {};
      \node[dot] at (V2) {};} +
    \tikz[baseline=\allinea]{
      \draw[middlearrow={latex}] (0,0) -- node[below]
      {$\psi_{0}$} (1.1,0) node[shape=coordinate] (V1) {};
      \draw (V1) node[above] {$V$} -- node[above,sloped] {$g$} ++(0.4,-0.2)
      node[shape=coordinate] (V2) {};
      \draw (V2) node[below] {$V$} -- node[below,sloped] {$g$} ++(0.6,0.3)
      node[shape=coordinate] (V3) {};
      \draw (V3) node[above] {$V$} -- node[above right,sloped] {$g$}
      ++(1.1,-0.4);
      \draw (1.7,0) circle (1);
      \node[dot] at (V1) {};
      \node[dot] at (V2) {};
      \node[dot] at (V3) {};} +\cdots $
  }
  \caption{Interpretazione fisica della serie di Born~\eqref{eq:serie-born}.  Il
    termine di ordine zero è la sola onda piana $\psi_{0}$ incidente, il termine
    di ordine $1$ è un'onda piana diffusa una volta, il termine di ordine $2$ è
    un'onda piana diffusa due volte, il termine di ordine $3$ è un'onda piana
    diffusa tre volte, ecc.}
\label{fig:serie-born}
\end{figure}
Riscriviamo schematicamente l'equazione di
Lippmann–Schwinger~\eqref{eq:lippmann-schwinger} in questo modo
\begin{equation}
  \psi = \psi_{0} + \int gV\psi,
\end{equation}
in cui per brevità abbiamo posto $g = 2mG/\hslash^{2}$.  Sostituiamo il secondo
membro sotto il segno di integrale
\begin{equation}
  \psi = \psi_{0} + \int gV\psi_{0} + \iint gVgV\psi.
\end{equation}
Possiamo continuare la procedura ottenendo
\begin{equation}
  \label{eq:serie-born}
  \psi = \psi_{0} + \int gV\psi_{0} + \iint gVgV\psi_{0} + \iiint gVgVgV\psi_{0}
  + \cdots.
\end{equation}
Questa è la \emph{serie di Born}.  Il termine $\psi_{0}$ di ordine zero della
serie rappresenta la funzione d'onda incidente $\psi_{0}$ non disturbata dal
potenziale, negli integrali successivi essa compare con un numero crescente di
potenze del prodotto $gV$ fra la funzione di Green e il potenziale di
interazione.  Nella figura~\ref{fig:serie-born} è rappresentata
l'interpretazione fisica della serie di Born: possiamo interpretare il termine
$\int gV\psi_{0}$ di ordine $1$ come l'onda piana $\psi_{0}$ che viene diffusa
una sola volta e poi si propaga liberamente, il termine $\iint gVgV\psi_{0}$ di
ordine $2$ come due processi di diffusione in sequenza, ecc.  In questo contesto
la funzione di Green è chiamata \emph{propagatore} perché spiega come l'onda
disturbata dal potenziale si propaga fra un'interazione e la successiva.  I
diagrammi di Feynman, espressi in termini di vertici ($V$) e propagatori ($g$),
alla base della formulazione della meccanica quantistica relativistica di
Feynman sono ispirati proprio alla serie di Born.

\subsection{Prima approssimazione di Born}
\label{sec:prima-approx-born}

Se il potenziale $V$ è sufficientemente piccolo, i termini della
serie~\eqref{eq:serie-born} contenenti potenze di $V$ di ordine $2$ o superiori
sono trascurabili rispetto al termine di ordine $1$ e arrestiamo la serie al
termine lineare in $gV$
\begin{equation}
  \label{eq:approx-born}
  \begin{split}
    \psi_{k}(\bm{r}) &= \psi_{0,k}(\bm{r}) + \int g(\bm{r},\bm{r}_{0};k)
    V(\bm{r}_{0}) \psi_{0,k}(\bm{r}_{0}) \dd^{3}\bm{r}_{0} \\
    &= \e^{\uimm \bm{k}\cdot\bm{r}} - \frac{\e^{\uimm kr}}{r}
    \frac{m}{2\pi\hslash^{2}}\int \e^{-\uimm (\bm{k}' - \bm{k})\cdot\bm{r}_{0}}
    V(\bm{r}_{0}) \dd^{3} \bm{r}_{0}.
  \end{split}
\end{equation}
Questa è l'\emph{approssimazione di Born}, chiamata a volte \emph{prima}
approssimazione di Born proprio perché la serie si arresta al termine di ordine
$1$.  È possibile migliorare l'approssimazione considerando anche termini
successivi nella serie.  La prima approssimazione di Born può essere espressa
dicendo che se il potenziale è tale da non modificare sostanzialmente la
funzione d'onda della particella incidente, nell'equazione~\eqref{eq:bar}
possiamo porre
$\psi_{k}(\bm{r}_{0}) \approx \psi_{0,k}(\bm{r}_{0}) =
\e^{\uimm \bm{k}\cdot\bm{r}_{0}}$.
In linea di principio diciamo che questa condizione è soddisfatta se il
potenziale è una debole perturbazione dell'hamiltoniana di particella libera
$H_{0} = -\hslash^{2}\nabla^{2}/2m$, ma l'applicabilità dell'approssimazione di
Born verrà discussa più estesamente nel paragrafo~\ref{sec:validita-born}.
L'ampiezza di diffusione di Born è, come al solito, il coefficiente di
$\e^{\uimm kr}/r$ nell'equazione~\eqref{eq:approx-born}
\begin{equation}
  \label{eq:ampiezza-born-braket}
  f_{k}(\theta,\phi) = -\frac{m}{2\pi\hslash^{2}} \braket{\psi_{0,\bm{k}'} | V |
    \psi_{0,\bm{k}}} = -\frac{m}{2\pi\hslash^{2}} \int \e^{-\bm{q}\cdot\bm{r}_{0}}
  V(\bm{r}_{0}) \dd^{3} \bm{r}_{0},
\end{equation}
con $\bm{q} = \bm{k}' - \bm{k}$ momento trasferito.  L'ampiezza di diffusione di
Born è quindi proporzionale alla trasformata di Fourier del potenziale fatta
rispetto al momento trasferito $\bm{q}$.

Per basse energie, cioè bassi valori di $k$, l'esponenziale nell'integrale è
essenzialmente costante nella regione di diffusione e l'ampiezza di diffusione
diventa
\begin{equation}
  \label{eq:ampiezza-born-basse}
  f_{k}(\theta,\phi) = -\frac{m}{2\pi\hslash^{2}} \int V(\bm{r}) \dd^{3}
  \bm{r}.
\end{equation}
Per semplicità abbiamo soppresso il pedice $0$ dalla variabile di integrazione
$\bm{r}_{0}$ dato che qui non c'è pericolo di ambiguità.

\begin{SCfigure}[2]
  \centering
  \begin{tikzpicture}[scale=2]
    \draw[-latex] (0,0) -- node[below] {$\bm{k} = k \versor{z}$} (2,0)
    node[shape=coordinate] (k) {};
    \draw[-latex] (0,0) -- node[sloped,above] {$\bm{k}' = k \versor{r}$}
    ($cos(30)*(2,0) + sin(30)*(0,2)$) node[shape=coordinate] (k1) {};
    \draw[-latex] (k) -- node[right] {$\bm{q} = \bm{k}' - \bm{k}$} (k1);
    \draw[-latex] (0.6,0) to[out=90,in=-60] node[right] {$\theta$}
    ($cos(30)*(0.6,0) + sin(30)*(0,0.6)$);
  \end{tikzpicture}
  \caption{Il momento trasferito $\bm{q} = \bm{k}' - \bm{k}$ è la base del
    triangolo isoscele di lati lunghi $k$ e con angolo al vertice $\theta$,
    quindi $q = \abs{\bm{q}} = 2k\sin(\theta/2)$.}
\label{fig:momento-trasferito}
\end{SCfigure}
Nel caso di potenziale a simmetria sferica, $V(\bm{r}) = V(r)$, scegliamo l'asse
$\versor{z}_{0}$ per la variabile di integrazione $\bm{r}_{0}$
nell'equazione~\eqref{eq:ampiezza-born-braket} lungo la direzione del momento
trasferito, in modo che $\bm{q}\cdot\bm{r}_{0} = qr_{0}\cos\theta_{0}$, così
\begin{equation}
  \label{eq:ampiezza-born-centrale}
  \begin{split}
    f_{k}(\theta) &= -\frac{m}{2\pi\hslash^{2}} \int_{0}^{2\pi}\dd\phi_{0}
    \int_{0}^{\infty}r_{0}^{2}\dd r_{0} V(r_{0}) \int_{-1}^{1}\e^{-\uimm q
      r_{0}\cos\theta_{0}} \dd(\cos\theta_{0}) \\
    &= -\frac{2m}{q\hslash^{2}}\int_{0}^{+\infty} rV(r)\sin(qr)\dd r.
  \end{split}
\end{equation}
Nel risultato finale abbiamo nuovamente soppresso il pedice $0$ dalla variabile
di integrazione.  Il modulo $q$ del momento trasferito vale
$q = 2k\sin(\theta/2)$, come mostrato nella figura~\ref{fig:momento-trasferito},
e la dipendenza angolare e dall'energia dell'ampiezza di diffusione è tutta
contenuta in $q$.  L'ampiezza di diffusione di Born per un potenziale centrale
dipende solo dalla colatitudine $\theta$, coerentemente con quanto detto nel
paragrafo~\ref{sec:sviluppo-onde}.  Poiché l'approssimazione di Born è
applicabile se il potenziale è sufficientemente debole, le deflessioni
$\delta_{l}$ prodotte dal potenziale di interazione sferico saranno
piccole,\footnote{Non è vero il contrario: si possono costruire potenziali che
  producono piccoli sfasamenti ma per i quali non è possibile applicare
  l'approssimazione di Born.  Le condizioni di validità dell'approssimazione
  sono discusse nel paragrafo~\ref{sec:validita-born}.}
% la nota precedente è riferita al primo esempio di "Surprises in Theoretical
% Physics", di Rudolf Peierls.
$\delta_{l} \ll 1$, e $\e^{2\uimm \delta_{l}} - 1 \approx 2\uimm \delta_{l}$.
L'ampiezza di diffusione~\eqref{eq:ampiezza-diffusione-op} si può allora
approssimare con
\begin{equation}
  f_{k}(\theta) \approx \frac{1}{k}\sum_{l}(2l+1) \delta_{l} P_{l}(\cos\theta).
\end{equation}
Apparentemente questo risultato è una contraddizione con il teorema ottico:
un'ampiezza di diffusione puramente reale implicherebbe una sezione d'urto
totale nulla.  Tuttavia dobbiamo ricordare che la prima approssimazione di Born
prende in considerazione solo il termine lineare del potenziale, mentre il
teorema ottico coinvolge la sezione d'urto totale che è un'espressione
quadratica dell'ampiezza di diffusione.  Per risolvere questo ``paradosso''
bisogna considerare le potenze di $V$ di ordine superiore a $1$ nell'espressione
dell'ampiezza di diffusione, che contribuiscono con termini complessi anche se
il potenziale è reale.  Per esempio si può verificare che l'ampiezza di
diffusione in seconda approssimazione di Born soddisfa il teorema ottico con la
sezione d'urto totale calcolata con la prima approssimazione di Born,
vedi~\textcite[361-362]{gottfried:quantum-mechanics}.  Il teorema ottico implica
allora che l'ampiezza di diffusione esatta contenga tutte le potenze di $V$
nella serie di Born, questo però solleva il problema della convergenza della
serie, problema di cui non ci occuperemo.
% TODO: eventualmente calcolare anche le onde parziali in approssimazione di
% Born.  Vedi, per esempio, Gottfried, pagina 361.

Nel limite di basse energie, cioè piccoli valori $k$ e quindi di $q$, possiamo
porre $\sin(qr)/q \approx r$ e l'ampiezza di diffusione di
Born~\eqref{eq:ampiezza-born-centrale} diventa
\begin{equation}
  \label{eq:ampiezza-born-centrale-basse}
  f_{k}(\theta) = f_{k} = -\frac{2m}{\hslash^{2}} \int_{0}^{\infty} r^{2}V(r)
  \dd r.
\end{equation}
Questa può essere ricavata anche dall'equazione~\eqref{eq:ampiezza-born-basse}
calcolando l'integrale per $V(\bm{r})$ = $V(r)$.  La diffusione da un potenziale
centrale nel limite di basse energie è approssimativamente isotropa e nello
sviluppo in onde parziali sono non trascurabili solo i primi termini, come
potevamo aspettarci dalle considerazioni fatte alla fine del
paragrafo~\ref{sec:sez-urto-op}.

Anche in approssimazione di Born vale la forma
asintotica~\eqref{eq:forma-asintotica} della funzione d'onda, allora la sezione
d'urto differenziale è il modulo quadro dell'ampiezza di diffusione,
$\sigma(\theta,\phi) = \abs{f_{k}(\theta,\phi)}^{2}$, e la sezione d'urto totale
è dato da
$\sigma_{\textup{tot}} = \int \sigma(\theta,\phi) \dd\Omega = \int
\abs{f_{k}(\theta,\phi)}^{2} \dd\Omega$.

\subsection{Validità dell'approssimazione}
\label{sec:validita-born}

In questo paragrafo determineremo delle condizioni che stabiliscono in quali
casi è possibile utilizzare l'approssimazione di Born.  Per semplicità di
calcoli assumiamo che qui il potenziale sia centrale: $V(\bm{r}) = V(r)$.
Riscriviamo la serie di Born~\eqref{eq:serie-born} più esplicitamente nel
seguente modo
\begin{equation}
  \label{eq:serie-born-esplicita}
  \begin{split}
    \psi_{k}(\bm{r}) &= \psi_{0,k}(\bm{r}) + \psi_{1,k}(\bm{r}) +
    \psi_{2,k}(\bm{r}) + \psi_{3,k}(\bm{r}) + \cdots \\
    &=\psi_{0,k} + \frac{2 m}{\hslash^{2}}\int\dd^{3}\bm{r}_{0}
    G_{+}(\bm{r},\bm{r}_{0};k) V(r_{0}) \psi_{0,k}(\bm{r}_{0}) \\
    &+ \bigg(\frac{2m}{\hslash^{2}}\bigg)^{2}
    \int\dd^{3}\bm{r}_{0}G_{+}(\bm{r},\bm{r}_{0};k)
    V(r_{0})\int\dd^{3}\bm{r}_{1} G_{+}(\bm{r}_{0},\bm{r}_{1};k) V(r_{1})
    \psi_{0,k}(\bm{r}_{1}) \\
    &+ \bigg(\frac{2m}{\hslash^{2}}\bigg)^{3}
    \int\dd^{3}\bm{r}_{0}G_{+}(\bm{r},\bm{r}_{0};k)
    V(r_{0})\int\dd^{3}\bm{r}_{1} G_{+}(\bm{r}_{0},\bm{r}_{1};k) V(r_{1}) \\
    &\cdot\int\dd^{3}\bm{r}_{2} G_{+}(\bm{r}_{1},\bm{r}_{2};k) V(r_{2})
    \psi_{0,k}(\bm{r}_{2}) + \cdots.
  \end{split}
\end{equation}
È possibile applicare la prima approssimazione di Born se
$\abs{\psi_{1,k}(\bm{r})} \ll \abs{\psi_{0,k}(\bm{r})} = 1$.  Ci aspettiamo che
l'onda di diffusione assuma i valori più grandi nelle vicinanze del centro
diffusore, cioè intorno al punto $\bm{r} = \bm{0}$, poiché lontano dall'origine
è fuori dall'influenza del potenziale e si comporta come un'onda sferica.
Pertanto valutiamo
\begin{equation}
  \label{eq:foobar}
  \begin{split}
    \psi_{1,k}(\bm{0}) &= -\frac{m}{2\pi\hslash^{2}}\int \frac{\e^{\uimm
        kr_{0}}}{r_{0}} V(r_{0}) \e^{\uimm \bm{k}\cdot\bm{r}_{0}}
    \dd^{3}\bm{r}_{0} \\
    &= -\frac{2m}{\hslash^{2}k} \int_{0}^{+\infty} \e^{\uimm kr_{0}}V(r_{0})
    \sin(kr_{0}) \dd r_{0}.
  \end{split}
\end{equation}
La quantità $F(k)=\e^{\uimm kr_{0}}\sin(kr_{0})/k$ assume il massimo valore in
modulo\footnote{Senza fare lo studio della funzione $F(k)$ è sufficiente
  osservare che $\abs{F(k)} = r_{0} \abs{\sin(kr_{0})/(kr_{0})}$ e l'andamento
  della funzione $\sin(x)/x$ è noto.}
per $k = 0$ e risulta $F(0) = \lim_{k \to 0} F(k) = r_{0}$.  Dunque troviamo una
prima condizione per l'applicabilità della prima approssimazione di Born per
qualunque valore di $k$ valutando l'espressione~\eqref{eq:foobar} per $k=0$,
cioè nel limite di basse energie, e imponendo che risulti
$\abs{\psi_{1,0}(\bm{0})} \ll 1$
\begin{equation}
  \frac{2m}{\hslash^{2}}\int_{0}^{+\infty}\abs{V(r_{0})}r_{0}\dd r_{0} \ll 1.
\end{equation}
Da qui si capisce perché l'approssimazione di Born possa essere applicata solo a
potenziali a corto raggio, cioè tali che $\lim_{r \to \infty}r^{2}V(r) =
0$.\footnote{A
  basse energie l'ampiezza di diffusione di Born è data
  dall'equazione~\eqref{eq:ampiezza-born-centrale-basse} e, affinché
  quell'integrale converga, il potenziale all'infinito deve tendere a zero
  ancora più rapidamente, in particolare deve risultare
  $\lim_{r \to \infty}r^{3}V(r) = 0$.}
Naturalmente il potenziale $V(r)$ deve avere anche un comportamento tale da
rendere l'integrale non divergente nell'origine, in particolare nell'origine può
tendere eventualmente all'infinito con un ordine minore o uguale ad $\alpha$,
con $\alpha \in \mathopen{]}0, 1\mathclose{[}$.  Per esempio, l'approssimazione
di Born è inapplicabile ad alcuni importanti potenziali che descrivono gli urti
fra atomi ma che per piccole distanze hanno andamenti del tipo $r^{-12}$.  Il
potenziale coulombiano, che va come $1/r$ e non è quindi un potenziale a corto
raggio, causa una divergenza nell'integrale precedente sia nell'origine sia
all'infinito.

Una condizione più stringente, spesso però troppo forte, può essere ricavata se
supponiamo che il potenziale abbia un valore massimo $V_{0}$ e raggio d'azione
$L$.  Sostituendo questi valori nell'equazione precedente abbiamo
\begin{equation}
  \label{eq:condizione-born}
  \frac{2m}{\hslash^{2}}\int_{0}^{L}\abs{V_{0}}r_{0}\dd r_{0} =
  \frac{\abs{V_{0}}mL^{2}}{\hslash^{2}} \ll 1 \iff \abs{V_{0}} \ll
  \frac{\hslash^{2}}{mL^{2}}.
\end{equation}
Possiamo dare un'interpretazione fisica alla quantità $\hslash^{2}/mL^{2}$: per
il principio di indeterminazione di Heisenberg questa è l'ordine di grandezza
dell'energia cinetica di una particella di massa $m$ confinata in un volume di
dimensione lineare $L$.  Quindi la condizione~\eqref{eq:condizione-born}
equivale a dire che l'approssimazione di Born è applicabile quando il potenziale
(se attrattivo) è sufficientemente debole da non creare uno stato legato per la
particella relativa.

Ad alte energia possiamo ricavare una condizione meno restrittiva.  Infatti, per
grandi valori di $k$ la quantità $F(k)$ nell'integrale
dell'equazione~\eqref{eq:foobar} diventa molto piccola per via del fattore $1/k$
e delle rapide oscillazioni del termine
$\e^{\uimm kr_{0}}\sin(kr_{0}) = (\e^{2\uimm kr_{0}} - 1)/2\uimm$.  In
particolare, facciamo nuovamente l'approssimazione che il potenziale abbia
valore massimo $V_{0}$ e raggio $L$, se risulta $kL \gg 1$ le oscillazioni
diventano molto frequenti all'interno del raggio d'azione del potenziale e
l'integrale
\begin{equation}
  \int_{0}^{L} V(r_{0}) \frac{\e^{2\uimm kr_{0}}}{2\uimm k} \dd r_{0} \sim
  \frac{V_{0}}{2\uimm k} \int_{0}^{L} \e^{2\uimm kr_{0}} \dd r_{0} = V_{0}
  \frac{1 - \e^{2\uimm kL}}{4k^{2}}
\end{equation}
è circa nullo.  Così troviamo la seguente condizione
\begin{equation}
  \abs{\psi_{1,k}(\bm{0})} \approx \frac{m}{\hslash^{2}k}\abs*{\int_{0}^{L}
    V(r_{0}) \dd r_{0}} \sim \frac{mL\abs{V_{0}}}{\hslash^{2}k} \ll 1 \iff
  \abs{V_{0}} \ll \frac{\hslash^{2}k}{mL} =
  \frac{\hslash^{2}}{mL^{2}} kL. % = E/kL.
\end{equation}
Come preannunciato, questa condizione, nel regime $kL \gg 1$, è meno restrittiva
della~\eqref{eq:condizione-born}, quindi se è possibili trattare un potenziale
come una perturbazione a basse energie è possibile farlo anche ad alte, ma non è
necessariamente vero il contrario.

Anche se non è semplice individuare una condizione precisa sotto la quale sia
possibile applicare l'approssimazione di Born, dai precedenti ragionamenti è
chiaro che questa diventa più affidabile per grandi valori dell'energia della
particella incidente ed è proprio in questi casi che è maggiormente utilizzata.
Bisogna però osservare che la condizione di energia del fascio incidente molto
maggiore del potenziale non è una condizione sempre sufficiente a giustificare
l'uso dell'approssimazione di Born.  Se il potenziale è debole ma ha un raggio
molto grande, la funzione d'onda rimarrà simile a un'onda piana nella regione di
influenza del potenziale, ma acquisterà nello stesso tempo un grosso sfasamento,
che non è compatibile con questa approssimazione.
% Ma questo è chiaro dalle due condizioni trovate (quella generale e quella per
% kL << 1): in ogni caso deve essere V_0 << (fattore)/L^2, quindi se aumenta il
% raggio L, il valore di V_0 deve diminuire di conseguenza.

\subsection{Esempi}
\label{sec:esempi-born}

\begin{figure}
  \centering
  \input{gnuplot/potenziale-yukawa}
  \caption{Andamento dei potenziali di Yukawa ($V_{\textup{Y}}(r) = \e^{-r}/r$)
    e di Coulomb ($V_{\textup{C}}(r)=1/r$).  Il potenziale di Yukawa va
    rapidamente a zero, mentre quello di Coulomb ha una coda molto lunga.}
  \label{fig:pot-yukawa}
\end{figure}
Calcoliamo l'ampiezza di diffusione di Born per il potenziale centrale di Yukawa
\begin{equation}
  V(r) = V_{0}\frac{\e^{-\mu cr/\hslash}}{r} = V_{0}\frac{\e^{-\alpha r}}{r},
\end{equation}
con $\mu$ massa della particella scambiata nell'interazione.  Per semplicità
abbiamo posto $\alpha = \mu c/\hslash$ e $1/\alpha$ rappresenta il raggio
d'azione.  Nella figura~\ref{fig:pot-yukawa} è rappresentato l'andamento del
potenziale di Yukawa.  Questo potenziale fu introdotto da Hideki Yukawa nel 1935
per descrivere le interazioni nucleari fra protoni e neutroni dovute allo
scambio di una particella massiva, che successivamente è stata identificata con
il pione.
% L'articolo di Yukawa è "On the Interaction of Elementary Particles. I",
% Proc. Phys. Math. Soc. Japan, 17, p. 48 (1935).  DOI:10.1143/PTPS.1.1.
% L'unica copia PDF non a pagamento che ho trovato è
% http://web.ihep.su/dbserv/compas/src/yukawa35/eng.pdf
Il comportamento del potenziale di Yukawa nell'origine e all'infinito soddisfa
le ipotesi di applicabilità dell'approssimazione di Born viste in precedenza.
Dall'equazione~\eqref{eq:ampiezza-born-centrale} abbiamo che l'ampiezza di
diffusione Born per il potenziale di Yukawa è
\begin{equation}
  f_{k}(\theta) = -\frac{2mV_{0}}{q\hslash^{2}} \int_{0}^{+\infty} \e^{-\alpha
    r}\sin(qr)\dd r = -\frac{2mV_{0}}{\hslash^{2}} \frac{1}{q^{2} + \alpha^{2}},
\end{equation}
con $q = 2k\sin(\theta/2)$.
\begin{figure}
  \centering
  \input{gnuplot/sez-urto-yukawa}
  \caption{Distribuzione angolare della diffusione elastica da potenziale di
    Yukawa in approssimazione di Born.  Nella figura è rappresentata la funzione
    $F(\theta) = 1/(4(k/\alpha)^{2}\sin^{2}(\theta/2)+1)^{2}$ per diversi valori
    di $(k/\alpha)^{2}$.}
  \label{fig:sez-urto-yukawa}
\end{figure}
La sezione d'urto differenziale è
\begin{equation}
  \label{eq:sezione-yukawa}
  \sigma(\theta) = \bigg(\frac{2mV_{0}}{\hslash^{2}}\bigg)^{2}
  \frac{1}{(4k^{2}\sin^{2}(\theta/2) + \alpha^{2})^{2}}.
\end{equation}
Questa sezione d'urto permette di evidenziare delle caratteristiche comuni a
molti tipi di potenziali.  Innanzitutto essa non dipende dal fatto che il
potenziale sia attrattivo o repulsivo poiché $V_{0}$ compare al quadrato.
Inoltre se $k/\alpha \ll 1$ la sezione d'urto è approssimativamente isotropa.
Per $k \to 0$ la sezione d'urto diventa indipendente dall'energia.  Invece
all'aumentare di $k$ la particella viene deflessa solo di angoli piccoli e la
sezione d'urto diventa sempre più piccata intorno alla direzione $\theta = 0$ e
si comporta come $1/q^{4}$, o $1/\theta^{4}$ per piccoli angoli.  Nella
figura~\ref{fig:sez-urto-yukawa} è rappresentata la distribuzione angolare della
sezione d'urto differenziale.  La sezione d'urto totale del potenziale di Yukawa
è
\begin{equation}
  \sigma_{\textup{tot}} = \int \sigma(\theta) \dd\Omega =
  \bigg(\frac{2mV_{0}}{\hslash^{2}}\bigg)^{2} \frac{4\pi}{4k^{2} + \alpha^{2}}.
\end{equation}
Questa va a $0$ per $k \to \infty$, caratteristica generale dei potenziali di
diffusione perché l'interazione diventa trascurabile rispetto all'energia
cinetica.  Ciò, però, non è più vero nella meccanica quantistica relativistica.

Abbiamo più volte notato che la teoria qui sviluppata non è applicabile, senza
opportune modifiche che però non
tratteremo,\footnote{Si può trovare la trattazione completa e rigorosa della
  diffusione dal potenziale di Coulomb
  in~\textcite[368-373]{gottfried:quantum-mechanics} e
  \textcites[655-659]{landau:meccanica-quantistica}.}
al potenziale coulombiano, il cui andamento è rappresentato nella
figura~\ref{fig:pot-yukawa}.  Tuttavia osserviamo che l'interazione
elettromagnetica è mediata dal fotone, che ha massa nulla, e il potenziale di
Yukawa tende a quello di Coulomb nel limite $\mu \to 0$.  Possiamo allora
provare a calcolare la sezione d'urto coulombiana facendo tendere $\alpha$ a $0$
nella formula~\eqref{eq:sezione-yukawa} e ponendo $V_{0} = Z_{1}Z_{2}e^{2}$
\begin{equation}
  \label{eq:sez-urto-coulomb}
  \sigma(\theta) =
  \bigg(\frac{2mZ_{1}Z_{2}e^{2}}{4\hslash^{2}k^{2}\sin^{2}(\theta/2)}\bigg)^{2}=
  \frac{(Z_{1}Z_{2}e^{2})^{2}}{16E^{2}\sin^{4}(\theta/2)}
\end{equation}
che è proprio la sezione d'urto di Rutherford.  La sezione d'urto totale è
infinita perché integrando questa sezione d'urto differenziale su tutto l'angolo
solido si ottiene una divergenza, a causa dell'andamento $\sin^{-4}(\theta/2)$.
Ciò è dovuto al fatto che il potenziale coulombiano non è a corto raggio.  In un
certo senso possiamo dire che nessuna particella (carica) riesce a sfuggire al
potenziale di Coulomb.  Nella pratica questo non si verifica e non è possibile
osservare la divergenza della sezione d'urto totale perché non è possibile
sottoporre effettivamente una particella a un singolo potenziale coulombiano.  È
interessante che la sezione per questo potenziale calcolata secondo le regole
della meccanica quantistica, sebbene per via
indiretta,\footnote{La sezione d'urto~\eqref{eq:sez-urto-coulomb} è valida anche
  nello studio ``esatto'' del potenziale coulombiano.}
sia uguale a quella ottenuta classicamente da Rutherford.  Questa coincidenza
può essere attribuita al fatto che il risultato non contiene $\hslash$, se si
identifica $\hslash^{2}k^{2}/2m$ con l'energia, quindi non cambia eseguendo il
passaggio al limite classico con $\hslash \to 0$.

\section{Operatori di diffusione}
\label{sec:operatori-diffusione}

Fino a qui abbiamo trattato la teoria della diffusione in meccanica quantistica
facendo uso della rappresentazione nella base delle coordinate.  In questo
paragrafo svilupperemo la teoria anche con la più elegante e generale notazione
operatoriale dei bra-ket che permette di svincolarsi da una particolare
rappresentazione.  Fra le altre cose, riotterremo l'equazione di
Lippmann–Schwinger e la serie di Born, valgono anche in questo caso le
considerazioni e interpretazioni di carattere fisico fatte precedentemente.

% Nota che per la definizione del risolvete ho seguito la convenzione del
% Ballentine: G = (z - H_0)^{-1}, mentre Martina e Boiti utilizzano la
% convenzione G = (H_0 - z)^{-1} e la matrice di trasferimento (vedi appunti del
% 27/01/2012) è scritta come G = G - G_0 T G_0
Consideriamo l'hamiltoniana $H$ nella forma
\begin{equation}
  H = H_{0} + V
\end{equation}
con $H_{0}$ hamiltoniana di particella libera e $V$ potenziale di diffusione.
Introduciamo gli operatori risolventi, rispettivamente, dell'hamiltoniana di
particella libera e dell'hamiltoniana completa
\begin{subequations}
  \begin{align}
    G_{0}(z) &= (z\mathbb{I} - H_{0})^{-1}, \\
    G(z) &= (z\mathbb{I} - H)^{-1},
  \end{align}
\end{subequations}
con $\mathbb{I}$ operatore identità e $z$, in generale, numero complesso.  Per
semplicità di scrittura, nel seguito ometteremo l'operatore $\mathbb{I}$.  I
risolventi non sono ben definiti nel campo dei numeri reali perché se $z$
appartiene allo spettro dell'operatore autoaggiunto $H_{0}$ o $H$, a seconda del
risolvente considerato, allora l'operatore $z - H_{0}$ o $z - H$ non è
invertibile.  Il valore del risolvente nel campo dei reali può essere ottenuto
mediante un passaggio al limite dal campo dei complessi.  Vedremo più avanti che
i risolventi corrispondono nel linguaggio operatoriale alle funzioni di Green.
Introduciamo l'\emph{operatore di trasferimento} $T$ definito dalla relazione
\begin{equation}
  \label{eq:oper-trasferimento}
  G(z) = G_{0}(z) + G_{0}(z) T(z) G_{0}(z).
\end{equation}
Dalla definizione di $T$ abbiamo
\begin{equation}
  \begin{split}
    T &= G_{0}^{-1} G G_{0}^{-1} - G_{0}^{-1} \\
    &= (z - H_{0})(GG_{0}^{-1} - 1) \\
    &= (z - H_{0})(GG_{0}^{-1} - GG^{-1}) \\
    &= (z - H_{0})GV,
  \end{split}
\end{equation}
in cui abbiamo sfruttato la relazione $G_{0}^{-1} - G^{-1} = V$.  D'altra parte
$G_{0}^{-1} G G_{0}^{-1} - G_{0}^{-1}$ può anche essere riscritto come
$(G_{0}^{-1}G - 1)(z - H_{0})$ e con calcoli analoghi si trova che
\begin{equation}
  T = VG(z - H_{0}).
\end{equation}
Confrontando i due risultati abbiamo
\begin{subequations}
  \begin{align}
    G_{0} T &= GV, \\
    T G_{0} &= VG.
  \end{align}
\end{subequations}
Sostituendo questo nella definizione~\eqref{eq:oper-trasferimento}
dell'operatore di trasferimento risulta
\begin{equation}
  \label{eq:baz}
  G(z) = G_{0}(z) + G(z)VG_{0}(z) = G_{0}(z) + G_{0}(z)VG(z).
\end{equation}
Questa equazione può essere risolta iterativamente ottenendo una serie
perturbativa formale
\begin{equation}
  \label{eq:oper-G-serie}
  G = G_{0} + G_{0}VG_{0} + G_{0}VG_{0}VG_{0} + \cdots
\end{equation}
chiamata \emph{serie di Born}.  Inoltre abbiamo
\begin{equation}
  \begin{split}
    T - V &= G_{0}^{-1} G V -V \\
    &= (G_{0}^{-1}G - 1)V \\
    &= (G_{0}^{-1} - G^{-1})GV \\
    &= VGV
  \end{split}
\end{equation}
da cui
\begin{equation}
  T = V + VGV.
\end{equation}
Sostituendo l'equazione~\eqref{eq:oper-G-serie} nella precedente troviamo
\begin{equation}
  \label{eq:oper-T-serie}
  T = V + VG_{0}V + VG_{0}V + VG_{0}VG_{0}V + \cdots.
\end{equation}
Nel paragrafo~\ref{sec:oper-ampiezza-diffusione} vedremo che, grazie a questo
risultato, la matrice di trasferimento può essere messa in relazione con
l'ampiezza di diffusione in approssimazione di Born.

\subsection{Equazione di Lippmann-Schwinger}
\label{sec:oper-equazione-di-ls}

Nella notazione bra-ket, l'equazione di Schrödinger stazionaria è
\begin{equation}
  H \ket{\psi_{a}^{(+)}} = E \ket{\psi_{a}^{(+)}},
\end{equation}
in cui $a$ indica l'insieme di numeri quantici, fra cui il vettore d'onda
$\bm{k}_{a}$, che caratterizzano lo stato e l'apice $(+)$ rappresenta che il
comportamento asintotico della funzione d'onda di diffusione deve essere del
tipo~\eqref{eq:forma-asintotica} cioè la somma di un'onda piana e un'onda
sferica \emph{uscente}.  È necessaria questa precisazione perché è possibile
considerare funzioni d'onda $\psi^{(-)}$ il cui comportamento asintotico è dato
dalla somma di un'onda piana e un'onda sferica \emph{entrante}.  Possiamo
riscrivere l'equazione di Schrödinger precedente anche come
\begin{equation}
  \label{eq:oper-schrodinger}
  (E - H_{0})\ket{\psi_{a}^{(+)}} = V \ket{\psi_{a}^{(+)}}.
\end{equation}
Da qui si vede che $\ket{\psi_{a}^{(+)}}$ è data dalla somma della quantità
$G_{0}(E^{+})V\ket{\psi_{a}^{(+)}}$, con
$G_{}(E^{+})
= \lim_{\epsilon \to 0+} G_{0}(E + \uimm \epsilon)$,
e di una funzione $\ket{\psi_{0}}$ tale che $(E - H_{0})\ket{\psi_{0}} = 0$,
cioè $\ket{\psi_{0}}$ deve essere autostato dell'hamiltoniana $H_{0}$ con lo
stesso autovalore di energia di $\ket{\psi_{a}^{(+)}}$.  Dunque abbiamo
\begin{equation}
  \label{eq:oper-lippmann-schwinger}
  \ket{\psi_{a}^{(+)}} = \ket{\psi_{0}} + G_{0}(E^{+})V\ket{\psi_{a}^{(+)}}
\end{equation}
e questa è l'\emph{equazione di Lippmann–Schwinger} nella notazione
operatoriale.  In rappresentazione di coordinate questa diventa proprio
l'equazione~\eqref{eq:lippmann-schwinger}.
L'equazione~\eqref{eq:oper-lippmann-schwinger} contiene ``più informazione''
della semplice equazioni di Schrödinger~\eqref{eq:oper-schrodinger} perché
incorpora la condizione al contorno data dal termine $\ket{\psi_{0}}$ che
descrive il comportamento della funzione d'onda in assenza di potenziale.  Se
poniamo nell'equazione di Lippmann–Schwinger $V = 0$ la soluzione sarà
$\ket{\psi_{0}}$, che è autostato dell'hamiltoniana di particella libera e
rappresenta il fascio incidente.  Quindi la forma dello stato di diffusione è
del tipo: fascio incidente $+$ onda diffusa dal potenziale.  Il risolvente
$G_{0}(E^{+})$ è associato a stati di diffusione $\ket{\psi^{(+)}}$
\emph{uscenti}, vedi~\textcite[450-452]{ballentine:quantum-mechanics}.  Un'altra
possibile scelta per il risolvente potrebbe essere
$G_{0}(E^{-}) = \lim_{\epsilon \to 0+} G_{0}(E - \uimm \epsilon)$, ma si può far
vedere che questa è associata a stati di diffusione $\ket{\psi^{(-)}}$
\emph{entranti} dati da
\begin{equation}
  \ket{\psi_{a}^{(-)}} = \ket{\psi_{0}} + G_{0}(E^{-})V\ket{\psi_{a}^{(-)}}.
\end{equation}
Avevamo incontrato una simile ambiguità anche per le funzioni di Green nel
paragrafo~\ref{sec:equazione-integrale} e anche in quel caso avevamo scelto la
funzione $G_{+}(\bm{r},\bm{r}';k)$ perché corrispondente a stati di diffusione
uscenti.  In effetti le funzioni di Green sono strettamente legate ai
risolventi.  Dalla definizione di $G_{0}(E^{\pm})$ abbiamo
\begin{equation}
  \begin{gathered}
    \braket{\bm{r} | (E - H_{0})G_{0}(E^{\pm}) | \bm{r}'} = \braket{\bm{r} |
      \bm{r}'} \iff \\
    \bigg(\frac{\hslash^{2}k^{2}}{2m} + \frac{\hslash^{2}}{2m}\nabla^{2}\bigg)
    \braket{\bm{r} | G_{0}(E^{\pm}) | \bm{r}'} = \delta(\bm{r} - \bm{r}')
  \end{gathered}
\end{equation}
e confrontando con l'equazione~\eqref{eq:green} riconosciamo che
\begin{equation}
  \braket{\bm{r} | G_{0}(E^{\pm}) | \bm{r}'} =
  \frac{2m}{\hslash^{2}}G_{\pm}(\bm{r},\bm{r}';k).
\end{equation}

Studiando l'evoluzione temporale delle funzioni d'onda $\psi_{a}^{(-)}$ si
scopre che rappresentano stati che per $t \to -\infty$ sono dati dalla
sovrapposizione di un'onda piana e un'onda di implosione verso il centro, invece
per $t \to +\infty$ si ha un'onda piana che si propaga nella direzione
$-\versor{k}_{a}$.  In definitiva, gli stati di diffusione entranti
$\psi_{a}^{(-)}$ rappresentano l'inversione temporale degli stati di diffusione
uscenti $\psi_{a}^{(+)}$.  Sebbene gli stati entranti siano una soluzione
matematicamente accettabile dell'equazione di Schrödinger, essi sono fisicamente
difficili da preparare, perché è necessario mantenere la coerenza della funzione
d'onda iniziale in una regione macroscopica.  È questo il motivo per il quale
utilizziamo l'insieme degli stati di diffusione uscenti come base completa degli
stati con energia positiva, invece degli stati di diffusione entranti.  L'intera
base dello spazio di Hilbert si ottiene aggiungendo gli eventuali stati legati
agli stati di diffusione.

Il prodotto scalare fra due stati di diffusione entrante e uscente
$\braket{\psi_{b}^{(-)} | \psi_{a}^{(+)}}$ associati a due insiemi di numeri
quantici $a$ e $b$ differenti (in particolare sono diversi i vettori d'onda
$\bm{k}_{a}$ e $\bm{k}_{b}$) definisce gli elementi della cosiddetta
\emph{matrice S} o \emph{matrice di diffusione} o \emph{matrice di scattering}
\begin{equation}
  S_{b,a} = \braket{\psi_{b}^{(-)} | \psi_{a}^{(+)}}.
\end{equation}

\subsection{Calcolo dell'ampiezza di diffusione}
\label{sec:oper-ampiezza-diffusione}

Mettiamo ora il relazione la matrice di trasferimento con l'ampiezza di
diffusione $f_{ab}^{(+)}(\Omega_{\bm{k}_{b}})$ che nella teoria generale della
diffusione si dimostra essere uguale a
(vedi~\textcite[436-441]{ballentine:quantum-mechanics})
\begin{equation}
  \label{eq:ampiezza-generale}
  f_{ab}^{(+)}(\Omega_{\bm{k}_{b}}) = -\frac{m}{2\pi\hslash^{2}}
  \braket{\psi_{0,b} | V | \psi_{a}^{(+)}},
\end{equation}
dove $a$ indica lo stato del fascio incidente (quindi l'onda incidente ha
vettore $\bm{k}_{a}$), $b$ quello dell'onda diffusa (che ha vettore d'onda
$\bm{k}_{b}$), $\Omega_{\bm{k}_{b}}$ è l'angolo di $\bm{k}_{b}$,
$\ket{\psi_{0,b}}$ è autostato dell'hamiltoniana di particella libera con
autovalore $E_{b} = \hslash^{2}k_{b}^{2}/2m$.  Avevamo trovato lo stesso
risultato anche nel paragrafo~\ref{sec:serie-born},
equazione~\eqref{eq:ampiezza-braket}, in cui abbiamo utilizzato una notazione
differente.  Dall'equazione~\eqref{eq:baz} abbiamo $G(z) = (1 + G(z)V)G_{0}(z)$,
da cui
\begin{equation}
  G(z)G_{0}^{-1}(z) = 1 + G(z)V.
\end{equation}
Dall'equazione~\eqref{eq:baz} abbiamo anche $G_{0}(z) = G(z) - G_{0}(z)VG(z) =
(1 - G_{0}(z)V)G(z)$, da cui
\begin{equation}
  G_{0}(z)G^{-1}(z) = 1 - G_{0}(z)V.
\end{equation}
Moltiplicando membro a membro le due equazioni sopra ricavate, da entrambi i
lati, abbiamo
\begin{equation}
  (1 + G(z)V)(1 - G_{0}(z)V) = (1 - G_{0}(z)V)(1 + G(z)V) = 1.
\end{equation}
Scriviamo l'equazione di Lippmann–Schwinger~\eqref{eq:oper-lippmann-schwinger}
come
\begin{equation}
  (1 - G_{0}(E^{+})V)\ket{\psi_{a}^{(+)}} = \ket{\psi_{0}}
\end{equation}
e sostituendo il risultato precedente abbiamo
\begin{equation}
  \label{eq:foobaz}
  \ket{\psi_{a}^{(+)}} = (1 + G(E^{+})V)\ket{\psi_{0}}.
\end{equation}
Abbiamo trovato che
$V\ket{\psi_{a}^{(+)}} = (V + VG(E^{+})V)\ket{\psi_{0}} =
T(E^{+})\ket{\psi_{0}}$. Gli elementi della matrice di trasferimento fra gli
stati $\ket{\psi_{0,\bm{k}_{b}}}$ e $\ket{\psi_{0,\bm{k}_{a}}}$, associati a
vettori d'onda, rispettivamente, $\bm{k}_{b}$ e $\bm{k}_{a}$, sono
\begin{equation}
  \label{eq:oper-T-amp}
  \begin{split}
    \braket{\psi_{0,\bm{k}_{b}} | T(E^{+}) | \psi_{0,\bm{k}_{a}}} &=
    \braket{\psi_{0,\bm{k}_{b}} | V | \psi_{a}^{(+)}} \\
    &= -\frac{2\pi\hslash^{2}}{m} f_{ab}^{(+)}(\Omega_{\bm{k}_{b}}),
  \end{split}
\end{equation}
o, equivalentemente,
\begin{equation}
  \label{eq:oper-amp-diff}
  f_{ab}^{(+)}(\Omega_{\bm{k}_{b}}) = -\frac{m}{2\pi\hslash^{2}}
  \braket{\psi_{0,\bm{k}_{b}} | T(E^{+}) | \psi_{0,\bm{k}_{a}}}.
\end{equation}
Abbiamo visto che l'operatore di trasferimento $T$ può essere espanso nella
serie~\eqref{eq:oper-T-serie}, quindi possiamo espandere anche l'ampiezza di
diffusione in una serie del tipo
\begin{equation}
  f_{ab} = \sum_{n = 1}^{\infty} f_{ab}^{(n)} = f_{ab}^{(1)} + f_{ab}^{(2)} +
  f_{ab}^{(3)} + \cdots.
\end{equation}
Infatti sostituendo la serie~\eqref{eq:oper-T-serie}
nell'espressione~\eqref{eq:oper-amp-diff} dell'ampiezza di diffusione abbiamo
\begin{equation}
  f_{ab} = -\frac{m}{2\pi\hslash^{2}} \braket{\psi_{0,\bm{k}_{b}} | (V +
    VG_{0}(E^{+})V + VG_{0}(E^{+})VG_{0}(E^{+})V + \cdots) |
    \psi_{0,\bm{k}_{a}}}.
\end{equation}
I primi termini della serie sono
\begin{subequations}
  \begin{align}
    f_{ab}^{(1)} &= -\frac{m}{2\pi\hslash^{2}}
    \braket{\psi_{0,\bm{k}_{b}} | V | \psi_{0,\bm{k}_{a}}}, \\
    \label{eq:ampiezza-born2-braket}
    f_{ab}^{(2)} &= -\frac{m}{2\pi\hslash^{2}} \braket{\psi_{0,\bm{k}_{b}} |
      VG_{0}(E^{+})V | \psi_{0,\bm{k}_{a}}}, \\
    f_{ab}^{(3)} &= -\frac{m}{2\pi\hslash^{2}} \braket{\psi_{0,\bm{k}_{b}} |
      VG_{0}(E^{+})VG_{0}(E^{+})V | \psi_{0,\bm{k}_{a}}}.
  \end{align}
\end{subequations}
In particolare, il termine $f_{ab}^{(1)}$ nella rappresentazione della base di
coordinate è esattamente l'ampiezza di
diffusione~\eqref{eq:ampiezza-born-braket} nella prima approssimazione di Born,
scritta con una notazione differente.  È facile verificare partendo dalla
serie~\eqref{eq:serie-born}, o dalla forma più
esplicita~\eqref{eq:serie-born-esplicita}, che i termini $f_{ab}^{(2)}$,
$f_{ab}^{(3)}$, ecc. sono le ampiezze di diffusione in seconda, terza
approssimazione di Born ecc.

La serie di Born per $\ket{\psi_{a}^{(+)}}$ può essere ottenuta risolvendo
iterativamente l'equazione di
Lippmann–Schwinger~\eqref{eq:oper-lippmann-schwinger}
\begin{equation}
  \ket{\psi_{a}^{(+)}} = \ket{\psi_{0}} + G_{0}(E^{+})V\ket{\psi_{0}} +
  G_{0}(E^{+})VG_{0}(E^{+})V\ket{\psi_{0}} + \cdots,
\end{equation}
oppure inserendo la serie di Born~\eqref{eq:oper-G-serie} del risolvente $G$
nell'equazione~\eqref{eq:foobaz}.  L'ampiezza di diffusione di Born poi si
ottiene sostituendo l'equazione precedente nella~\eqref{eq:oper-T-amp}.

\phantomsection
\addcontentsline{toc}{section}{\refname}
\printbibliography

\end{document}

%%% Local Variables:
%%% mode: latex
%%% TeX-master: t
%%% End:
